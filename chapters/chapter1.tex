% introduction.tex

\chapter{Principi di Fluidodinamica ed Elasiticità}

% $$\ps{f}{v}[H][A]$$

\section{Le coordinate euleriane e le coordinate lagrangiane, configurazione di riferimento e deformazione}

Sia $\Omega \subset \mathbb{R}^{2,3}$ una regione dello spazio contenente un corpo continuo, allora $\mathbf{x} \in \Omega$ è un punto in coordinate euleriane e rappresenta un punto fisso nello spazio, detto punto spaziale o coordinata euleriana. 
Sia $\omega \subset \mathbb{R}^{2,3}$ la configurazione di riferimento del corpo in esame, allora $\mathbf{s} \in \omega$ è un punto in coordinate lagrangiane e rappresenta un punto ben definito del corpo, detto punto materiale o coordinata lagrangiana.
La relazione tra i due sistemi di rifemento è data dalle funzioni
\begin{equation*}  
\mathbf{X}:\omega \times [0,T] \to \Omega
\end{equation*}
\begin{equation*}  
\mathbf{q}:\Omega \times [0,T] \to \omega
\end{equation*}

Osserviamo che $\mathbf{X}(\mathbf{s},t)$ rappresenta la traiettoria del punto materiale $\mathbf{s}$.

Osserviamo inoltre che valgono le seguenti uguaglianze
\begin{equation*}  
\mathbf{x} = \mathbf{X}(\mathbf{q}(\mathbf{x},t),t)
\end{equation*}
\begin{equation*}  
\mathbf{s} = \mathbf{q}(\mathbf{X}(\mathbf{s},t),t)
\end{equation*}
e assumiamo che per ogni istante $t$ le due funzioni $\mathbf{X}$ e $\mathbf{q}$ siano inveribili e lipshitziane.

Definiamo il gradiente di deformazione $\mathbb{F}$ come
\begin{equation*}  
\mathbb{F}_{\alpha i} = (\Nabla_{s}\mathbf{X}(\mathbf{s},t))_{\alpha i} = \frac{\partial X_{\alpha}(\mathbf{s},t)}{\partial s_{i}}
\end{equation*}

L'ipotesi che $\mathbf{X}$ sia invertibile implica che $\mathbb{F}$ abbia determinante diverso da zero e quindi assumendo che, a meno di cambiamento di coordinate, sia positivo al tempo $t=0$ rimarrà positivo per ogni tempo $t>0$.

Il valore di $|\mathbb{F}|$ rappresenta il rapporto tra la misura della regione attorno al punto $\mathbf{s}$ e la misura della regione attorno al punto $\mathbf{X}(\mathbf{s},t)$.

La derivata temporale nei due riferimenti.

\subsection{Il teorema del trasporto}

Il teorema del trasporto di Reynolds permette di studiare l'evoluzione delle quantità fisiche di una regione di corpo in movimento.

\begin{theorem}
Dato un campo scalare o vettoriale $\Phi(\mathbf{x},t)$ e una regione regolare di materiale $\mathcal{P}\in\omega$, sia $\mathcal{P}_{t} = \mathbf{X}(\mathcal{P},t)$ la stessa regione in coordinate euleriane al tempo $t$ allora per ogni $t\in ]0,T[$ vale:
\begin{equation}
\frac{d}{dt}\int_{\mathcal{P}_{t}}\Phi(\mathbf{x},t)\,d\mathbf{x} = \int_{\mathcal{P}_{t}}(\frac{D\Phi(\mathbf{x},t)}{Dt}+\Phi(\mathbf{x},t)\Nabla\cdot\mathbf{u}(\mathbf{x},t))\,d\mathbf{x}
= \int_{\mathcal{P}_{t}}\frac{\partial\Phi(\mathbf{x},t)}{\partial t}+ \int_{\partial\mathcal{P}_{t}}\Phi(\mathbf{x},t)\mathbf{u}(\mathbf{x},t)\cdot\mathbf{n}\,d\mathbf{x}
\end{equation}
\end{theorem}
\begin{lemma}
Sia $J(\mathbf{x},t)=|\mathbb{F}|(\mathbf{x},t)$ allora
\begin{equation}
\frac{\partial}{\partial t}J = J (\Nabla \cdot \mathbf{u} )
\end{equation}
\end{lemma}
\begin{proof}
Osserviamo che 
\begin{equation}
\frac{\partial}{\partial t}\mathbf{X}(\mathbf{x},t) = \mathbf{u}(\mathbf{X}(\mathbf{x},t))
\end{equation}
Per le proprietà del determinante si ha che
\begin{equation}
\frac{\partial}{\partial t}J = ...
\end{equation}
e quindi
\begin{equation}
\frac{\partial}{\partial t}\frac{\partial J_x}{\partial x} =
\frac{\partial}{\partial x}\frac{\partial J_x}{\partial t} = 
\frac{\partial}{\partial x}u(\mathbf{x},t)
\end{equation}
e 
\begin{equation}
\frac{\partial u}{\partial x} J + \frac{\partial u}{\partial y} J +\frac{\partial u}{\partial z} J = (\Nabla \cdot \mathbf{u}) J
\end{equation}
\end{proof}
\begin{proof}[Dimostrazione (Teorema del trasporto)]
Sfruttando il lemma, effettuendo un cambiamento di variabile e differenziando sotto il segno di integrale, osservato che $J$ rappresenta il fattore introdotto dal cambio di variabile e che il dominio di integrazione non dipende più da $t$ si ottiene la tesi.
\end{proof}

A partire dal lemma si può ricavare un espressione in coordinate euleriane del vincolo di incomprimibilità
\begin{equation}
0 = \frac{d}{dt}\text{volume}(\mathcal{W}_t) = \frac{d}{dt}\int_{\mathcal{W}_t}\,dV = \int_{\mathcal{W}_t} \Nabla \cdot \mathbf{u}\,dV
\end{equation}
e questo implica che $\Nabla \cdot \mathbf{u} = 0$

\section{Cenni di Fluidodinamica}

%A Mathematical Introduction to Fluid Mechanics, A. Chorin, J. E. Marsden, 1992 Springer-Verlag Publishing Company Inc., terza edizione (forse preprint quarta) Capitolo 1 1-46
La fluidodinamica studia e descrive il movimento dei fluidi. Le equazioni che governano questo movimento sono ricavate a partire dalle leggi di conservazione della massa, del momento e dell'energia. Nel caso di fluidi perfetti si ottengono le equazioni di Eulero, mentre nel caso di fluidi viscosi si ottengono le equazioni di Navier-Stokes.

\subsection{Le equazioni di Eulero}

Sia $\mathcal{D} \subseteq \mathbb{R}^{2,3}$ una porzione di spazio contenente un fluido, $\mathbf{x} \in \mathcal{D}$ un punto del fluido in coordinate euleriane, $\mathbf{u}(\mathbf{x},t)$ la velocità del fluido nel punto $\mathbf{x}$ al tempo $t$. Per ogni tempo $t$, $\mathbf{u}$ è un campo vettoriale su $\mathcal{D}$ ed è detto campo delle velocità del fluido. Sia $\rho(\mathbf{x},t)$ la densità del fluido nel punto $\mathbf{x}$ al tempo $t$.  L'esistenza di $\rho$ è detta ipotesi del continuo ed equivale a considerare il corpo come un oggetto senza soluzione di continuità trascurando la natura molecolare del fluido. Questa ipotesi rimane accettabile fintanto che si studiano i fenomeni macroscopici che avvengono nel fluido. Sia $\mathcal{W} \subseteq \mathcal{D}$ una parte dello spazio occupato dal fluido e $m(\mathcal{W},t)= \int_{\mathcal{W}}\rho(\mathbf{x},t) \, dV$ la massa del fluido contenuto in $\mathcal{W}$ al tempo $t$. Si supponga sempre inoltre che le funzioni usate abbiano sufficiente regolarità per dare senso alle espressioni scritte. 

Le equazioni di Eulero sono ricavate a partire del principio di conservazione della massa (\emph{la massa non si crea e non si distrugge}), del momento (\emph{seconda legge di Newton}) e dell'energia (\emph{l'energia non si crea e non si distrugge}).

\paragraph{La conservazione della massa}
Nell'ipotesi che la massa del fluido si conservi è necessario che le uniche variazioni di massa per ogni porzione di fluido siano dovute al solo fluido che entra o esce dalla data porzione. Sia $\partial \mathcal{W}$ il bordo, supposto liscio, di  $\mathcal{W}$ e $\mathbf{n}(\mathbf{x})$ la normale uscente a $\partial \mathcal{W}$ nel punto $\mathbf{x}$ allora la massa del fluido uscente da $\mathcal{W}$ per unità di area $dA$ per unità di tempo è data da $\rho \mathbf{u}\cdot \mathbf{n}$.
\begin{equation*}  
\begin{split}
\frac{d}{dt} m(\mathcal{W},t) &= \frac{d}{dt}  \int_{\mathcal{W}}\rho(\mathbf{x},t) \, dV \\
                              &= \int_{\mathcal{W}}\frac{\partial}{\partial t}\rho(\mathbf{x},t) \, dV \\
                              &= - \int_{\partial \mathcal{W}}\rho(\mathbf{x},t) \mathbf{u}(\mathbf{x},t) \cdot \mathbf{n}(\mathbf{x}) \, dV
\end{split}
\end{equation*}
che grazie al teorema della divergenza diventa
\begin{equation*}
\int_{\mathcal{W}}\frac{\partial}{\partial t}\rho(\mathbf{x},t) \, dV = -\int_{\mathcal{W}}\Nabla \cdot \left (\rho(\mathbf{x},t) \mathbf{u}(\mathbf{x},t) \right ) \, dV
\end{equation*}
e quindi
\begin{equation}\label{eq:conservazione massa integrale}
\boxed{
\int_{\mathcal{W}}\left [ \frac{\partial}{\partial t}\rho(\mathbf{x},t) + \Nabla \cdot (\rho(\mathbf{x},t) \mathbf{u}(\mathbf{x},t)) \right ]\, dV = 0
}
\end{equation}
L'equazione \ref{eq:conservazione massa integrale} è detta legge della conservazione della massa in forma integrale.

Dovendo essere vera per ogni $\mathcal{W}$ l'equazione \ref{eq:conservazione massa integrale} è equivalente a 
\begin{equation}\label{eq:conservazione massa differenziale}
\boxed{
\frac{\partial \rho}{\partial t} + \Nabla \cdot (\rho \mathbf{u}) = 0
}
\end{equation}
Questa è la forma differenziale della legge di conservazione della massa, altrimenti detta equazione di continuità.

Lo stesso risultato si può ottenere applicando il teorema del trasporto
\begin{equation}
0 = \frac{d}{dt}\int_{\mathcal{W}_t}\rho\,dV = \int_{\mathcal{W}_t}(\frac{D\rho}{dt}+\rho(\Nabla\cdot\mathbf{u}))\,dV
\end{equation}


\paragraph{La conservazione del momento}
Sia $\mathbf{x}(t) = (x(t),y(t),z(t))$ la traiettoria seguita da una particella del fluido e $\mathbf{u}(t) = (\dot{x}(t),\dot{y}(t),\dot{z}(t)) = \mathbf{u}(\mathbf{x}(t)) = \frac{d\mathbf{x}}{dt}(t)$ la sua velocità allora l'accelerazione della particella sarà data da
\begin{equation*} 
\mathbf{a}(\mathbf{x}(t)) = \frac{d^2\mathbf{x}}{dt^2} (t) = \frac{\partial  \mathbf{u}}{\partial x}\dot{x} + \frac{\partial  \mathbf{u}}{\partial y}\dot{y} + \frac{\partial  \mathbf{u}}{\partial z}\dot{z} + \frac{\partial \mathbf{u}}{\partial t}
= \nabla  \mathbf{u} \cdot  \mathbf{u} + \frac{\partial \mathbf{u}}{\partial t} = \frac{D \mathbf{u}}{Dt}
\end{equation*}
dove $\frac{D}{Dt}=\partial_t + \mathbf{u}\cdot\Nabla$ è detta derivata materiale [...].
Lo stesso ragionamento è applicabile a ogni funzione $f=f(x(t),y(t),z(t),t)$ scalare o vettoriale
\begin{equation*}
\frac{df}{dt}=\partial_tf+\mathbf{u}\cdot\Nabla f = \frac{Df}{Dt}
\end{equation*}

Le forze che agiscono su di un corpo continuo possono essere di due soli tipi: le forze di stress e le forze esterne o esogene. Le forze di stress sono le forze applicate su una superficie interna al corpo dal resto del corpo. Le forze esterne sono invece forze applicate per unità di volume da agenti esterni al corpo[...]. Un fluido si dice ideale se esiste una funzione $p(\mathbf{x},t)$ detta pressione per cui le forze di stress sono del tipo $\mathbf{F}\,dA = p(\mathbf{x},t)\mathbf{n}$ cioè sono forze normali alla superficie. In particolare l'assenza di componenti tangenziali delle forze di stress impedisce l'instaurarsi o il cessare di movimenti rotatori nel fluido.

Presa $\mathcal{W}$ una regione del fluido la forza di stress applicata dal resto del corpo è
$$\mathbf{S}_{\partial\mathcal{W}} = - \int_{\partial\mathcal{W}} p\mathbf{n}\,dA$$.
Scelta una direzione $\mathbf{e}$ applicando il teorema della divergenza
$$\mathbf{e}\cdot\mathbf{S}_{\partial\mathcal{W}} = - \int_{\partial\mathcal{W}} p\mathbf{e}\cdot\mathbf{n}\,dA
                                                  = - \int_{\mathcal{W}} \nabla p \cdot \mathbf{e} \, dV $$
e quindi
$$\mathbf{S}_{\partial\mathcal{W}} = - \int_{\mathcal{W}} \nabla p \, dV $$

Usando la seconda legge di Newton e aggiungendo il termine relativo alle forze esterne si ottiene 
\begin{equation}\label{eq:conservazione momento differenziale}
\boxed{
\rho \frac{\partial \mathbf{u}}{\partial t} = - \rho \mathbf{u} \cdot \nabla \mathbf{u} - \nabla p + \rho \mathbf{b}
}
\end{equation}
che è la legge di conservazione del momento in forma differenziale.

La forma integrale della legge di conservazione del momento può essere ricavata direttamente dalla forma differenziale[...]
\begin{equation}
\boxed{
\frac{d}{dt}\int_{\mathcal{W}}\rho \mathbf{u} = - \int_{\partial\mathcal{W}}(p\mathbf{n}+\rho\mathbf{u}(\mathbf{u} \cdot \mathbf{n})) \, dA + \int_{\mathcal{W}}\rho\mathbf{b}\, dV
}  
\end{equation}

E' possibile ricavare la forma integrale della legge di conservazione del momento senza passare dalla forma differenziale.
Sia $\phi(\mathbf{x},t)$ la traiettoria seguita dal punto materiale che si trova in posizione $\mathbf{x}$ al tempo $t=0$. Si ipotizzi che $\phi$ sia regolare e invertibile per $t$ fissato. Sia $\phi_t:\mathbf{x}\mapsto\phi (\mathbf{x},t)$ la mappa che associa ad ogni punto la posizione che avrà al tempo $t$. $\phi$ è detta mappa del flusso del fluido. Indichiamo con $\mathcal{W}_t = \phi_t(\mathcal{W})$ il volume $\mathcal{W}$ trasportato dal fluido.

La legge di conservazione del momento si scrive quindi come
\begin{equation}\label{eq:conservazione momento integrale}
\frac{d}{dt} \int_{\mathcal{W}_t} \rho \mathbf{u} \, dV = S_{\partial \mathcal{W}_t} +  \int_{\mathcal{W}_t} \rho \mathbf{b} \, dV
\end{equation}


Questa è equivalente alla forma differenziale se le funzioni sono sufficientemente regolari.

Compiendo un cambio di variabile nell'integrale
$$\frac{d}{dt} \int_{\mathcal{W}_t} \rho \mathbf{u} \, dV = \frac{d}{dt} \int_{\mathcal{W}} (\rho \mathbf{u})(\phi(\mathbf{x},t),t) J(\mathbf{x},t) \, dV$$
dove $J(\mathbf{x},t)$ è il determinante della matrice jacobiana di $\phi_t$.

Differenziando ora sotto il segno di integrale nel secondo membro
\begin{equation*}  
\begin{split}
\frac{d}{dt} \int_{\mathcal{W}_t} \rho \mathbf{u} \, dV &= \int_{\mathcal{W}} \frac{\partial}{\partial t}[ (\rho \mathbf{u})(\phi(\mathbf{x},t),t) J(\mathbf{x},t) ]\, dV \\
&= \int_{\mathcal{W}} \left[\frac{\partial}{\partial t} (\rho \mathbf{u})(\phi(\mathbf{x},t),t) J(\mathbf{x},t) + (\rho \mathbf{u})(\phi(\mathbf{x},t),t) \frac{\partial}{\partial t}  J(\mathbf{x},t)\right] \, dV
\end{split}
\end{equation*}

Si osservi che $$\frac{\partial}{\partial t} (\rho \mathbf{u})(\phi(\mathbf{x},t),t) = \left(\frac{D}{D t} \rho \mathbf{u}\right)(\phi(\mathbf{x},t),t) $$.

Enunciamo ora il seguente lemma
\begin{lemma}\label{lemma:incomprimibile}
$\frac{\partial}{\partial t}  J(\mathbf{x},t) = J(\mathbf{x},t) \Nabla \cdot \mathbf{u} (\phi(\mathbf{x},t),t)$  
\end{lemma}
\begin{proof}
[...]
\end{proof}

Applicando il lemma si ottiene
\begin{equation*}
\begin{split}
\frac{d}{dt}\int_{\mathcal{W}_t}\rho\mathbf{u}\, dV &= \int_{\mathcal{W}} \left \{ \left ( \frac{D}{Dt}\rho\mathbf{u} \right )(\phi(\mathbf{x},t),t)+(\rho\mathbf{u})(\Nabla\cdot\mathbf{u})(\phi(\mathbf{x},t),t) \right \} J(\mathbf{x},t)\, dV\\
&=\int_{\mathcal{W}_t}\left \{ \frac{D}{Dt}(\rho\mathbf{u})+(\rho\Nabla\cdot\mathbf{u})\mathbf{u}\right \}\,dV
\end{split}
\end{equation*}
Usando ora l'equazione \ref{eq:conservazione massa differenziale}
$$\frac{D}{Dt}\rho+\rho\Nabla\cdot\mathbf{u}=\frac{\partial\rho}{\partial t}+\Nabla\cdot(\rho\mathbf{u})$$
e quindi
$$\frac{d}{dt}\int_{\mathcal{W}_t}\rho\mathbf{u}\, dV = \int_{\mathcal{W}_t}\rho\frac{D\mathbf{u}}{Dt}\, dV$$.

Questo ragionamento è in realtà valido per ogni funzione e prende il nome di teorema del trasporto.
\begin{theorem}{Teorema del trasporto}
$$ \forall f:\mathbb{R}^3\times[0,T]\mapsto M $$ $$\frac{d}{dt}\int_{\mathcal{W}_t}\rho f\, dV = \int_{\mathcal{W}_t}\rho\frac{Df}{Dt}\,dV$$
\end{theorem}

Essendo le funzioni continue e $\mathcal{W}$ qualunque allora \ref{eq:conservazione momento integrale} è equivalente a \ref{eq:conservazione momento differenziale}.

Sfruttando il lemma \ref{lemma:incomprimibile}possiamo caratterizzare i fluidi incomprimibili infatti
$$\volume(\mathcal{W}_t) = \int_{\mathcal{W}_t} \, dV$$
quindi per un fluido incomprimibile
$$0 = \frac{d}{dt} \int_{\mathcal{W}_t} \, dV  = \int_{\mathcal{W}_t} \Nabla \cdot \mathbf{u} \, dV$$
Quindi un fluido è incomprimibile se e solo se $\Nabla \cdot \mathbf{u}=0$ ( o anche se e solo se $J \equiv 1$ ).
$$0 = \frac{\partial \rho}{\partial t} + \Nabla\cdot(\rho \mathbf{u}) = \frac{D\rho}{Dt} + \rho\Nabla\cdot \mathbf{u}$$
quindi un fluido è incomprimibile se e solo se $\frac{D \rho}{Dt} = 0$ cioè se e solo se la densità del fluido è costante seguendo il flusso.
Un fluido si dice omogeneo se $\rho(\mathbf{x},t)=C_t \quad \forall \mathbf{x}$, è anche incomprimibile se $C_t=C \quad \forall t$

Diamo ora un'altra caratterizzazione della legge di conservazione della massa sfruttando il teorema del trasporto.
$$\frac{d}{dt}\int_{\mathcal{W}_t}\rho\,dV=0$$
e quindi
$$\int_{\mathcal{W}_t}\rho(\mathbf{x},t)\,dV=\int_\mathcal{W}\rho(\mathbf{x},0)\,dV$$
Effettuando un cambio di variabile
$$\int_{\mathcal{W}}\rho(\phi(\mathbf{x},t),t)J(\mathbf{x},t)\,dV=\int_\mathcal{W}\rho(\mathbf{x},0)\,dV$$
e quindi
\begin{equation}
  \boxed{
  \rho(\phi(\mathbf{x},t),t)J(\mathbf{x},t)=\rho(\mathbf{x},0)\,dV
  }
\end{equation}
In particolare si nota che un fluido omogeneo ma non incomprimibile non rimane omogeneo col passare del tempo.

\paragraph{La conservazione dell'energia}
Sia $$E_{\text{kinetic}} = \frac{1}{2} \int_{\mathcal{W}} \rho \Vert \mathbf{u} \Vert^2 \, dV$$ l'energia cinetica del fluido contenuto in $\mathcal{W}$, $E_{\text{internal}}$ l'energia interna dovuta a fenomeni microscopici tra le molecole del fluido e $$E_{\text{total}}=E_{\text{kinetic}}+E_{\text{internal}}$$ l'energia totale.
La variazione dell'energia cinetica nella porzione di fluido in movimento $\mathcal{W}_t$ calcolata grazie al teorema del trasporto è 
\begin{equation}
\begin{split}
\frac{d}{dt} E_{\text{kinetic}} &= \frac{d}{dt} \frac{1}{2} \int_{\mathcal{W}} \rho \Vert \mathbf{u} \Vert^2 \, dV \\
                                &= \frac{1}{2}  \int_{\mathcal{W}_t} \rho \frac{D\Vert \mathbf{u} \Vert^2}{Dt} \, dV \\
                                &=  \int_{\mathcal{W}_t} \rho ( \mathbf{u} \cdot ( \frac{\partial \mathbf{u}}{\partial t} + (\mathbf{u} \cdot \nabla) \mathbf{u} )) \, dV
\end{split}
\end{equation}
[...]

Consideriamo ora il caso in cui il fluido sia incomprimibile e che tutta l'energia del fluido sia nella forma di energia cinetica (oppure è sufficiente che l'energia interna sia costante). Allora la variazione dell'energia cinetica è dovuta al lavoro compiuto dalla pressione e dalle forze esterne

$$\frac{d}{dt} E_{\text{kinetic}} = - \int_{\partial \mathcal{W}_t} p \mathbf{u} \cdot \mathbf{n} \, dA + \int_{\mathcal{W}_t} \rho \mathbf{u} \cdot \mathbf{b} \, dV $$

e quindi usando il teorema della divergenza e tenendo conto che $\nabla \cdot \mathbf{u} = 0$

\begin{equation}
\begin{split}
\int_{\mathcal{W}_t} \rho \left ( \mathbf{u} \cdot \left ( \frac{\partial \mathbf{u}}{\partial t} + \mathbf{u} \cdot \Nabla \mathbf{u} \right )\right ) \, dV
 &= - \int_{\mathcal{W}_t} ( \Nabla \cdot (p\mathbf{u}) - \rho \mathbf{u} \cdot \mathbf{b}) \, dV \\
 &=  - \int_{\mathcal{W}_t} ( \nabla p  \cdot \mathbf{u}) - \rho \mathbf{u} \cdot \mathbf{b}) \, dV
\end{split}
\end{equation}
Osserviamo che se varia solo l'energia cinetica allora il fluido è incomprimibile [oppure $p=0$, fluido fermo].
 
Si ottengono cosi le equazioni di Eulero per i fluidi incomprimibili
\begin{equation}
\boxed{
\begin{aligned}
\rho \frac{D\mathbf{u}}{Dt} &= - \nabla p + \rho \mathbf{b} \\
\frac{D \rho }{Dt} &= 0 \\
\nabla \cdot \mathbf{u} &= 0
\end{aligned}
}
\end{equation}
a cui va aggiunta la condizione al bordo 
$$\mathbf{u} \cdot \mathbf{n} = 0 \quad \text{su} \quad \partial \mathcal{D}$$
 
[FLUIDI ISENTROPICI?]
Consideriamo il caso ora di un fluido isoentropico ovvero in cui esiste una funzione $w$ chiamata entalpia per cui
$\nabla w = \frac{1}{\rho} \nabla p$
[parentesi termodinamica]
La maggior parte dei gas reali può essere considerata come un fluido isoentropico in cui $p = A \rho ^ \gamma$ con $A$ e $\gamma \geq 1$ costanti.
 
\subsection{Rotazioni e vorticità}

Sia $\mathbf{u} = (u,v,w)$ il campo delle velocità di un fluido allora il suo rotore 
$$\boldsymbol{\xi} = \nabla \times \mathbf{u} = (\partial_y w - \partial_z v, \partial_z u - \partial_x w, \partial_x v - \partial_y u)$$
è detto campo della vorticità del fluido.

Si può dimostrare che $\mathbf{u}$ è localmente la somma di una traslazione rigida, di una deformazione e di una rotazione rigida di vettore $\frac{\boldsymbol{\xi}}{2}$. Cominciamo enunciando il seguente teorema che vale per ogni campo vettoriale definito su $\mathbb{R}^3$.

\begin{theorem}
Sia $\mathbf{x} \in \mathbb{R}^3$ e $\mathbf{y} = \mathbf{x} + \mathbf{h}$ un punto vicino. Allora 
\begin{equation}\label{eq:helmoltz}
\mathbf{u}(\mathbf{y}) = \mathbf{u}(\mathbf{x}) + \mathbf{D}(\mathbf{x}) \cdot \mathbf{h} + \frac{1}{2} \boldsymbol{\xi}(\mathbf{x}) \times \mathbf{h} + \mathcal{O} ( h^2)
\end{equation}
con $\mathbf{D}$ matrice simmetrica.
\end{theorem}
\begin{proof}
Sia
\begin{equation}
\Nabla\mathbf{u}={\setlength\arraycolsep{2pt} 
\left(\begin{array}{ccc} 
\partial_x u&  \partial_y u &\partial_z u\\ 
\partial_x v&  \partial_y v &\partial_z v\\ 
\partial_x w&  \partial_y w &\partial_z w\\
\end{array}\right) 
}
\end{equation}
la matrice jacobiana di $\mathbf{u}$.
Per il teorema di Taylor
$$\mathbf{u}(\mathbf{y}) = \mathbf{u}(\mathbf{x}) + \Nabla\mathbf{u}(\mathbf{x}) \cdot \mathbf{h} + \mathcal{O} ( h^2)$$
Siano rispettivamente
$$\mathbf{D} = \frac{1}{2} [ \Nabla\mathbf{u} + \Nabla\mathbf{u} ^T]$$
la parte simmetrica e
$$\mathbf{S} = \frac{1}{2} [ \Nabla\mathbf{u} - \Nabla\mathbf{u} ^T]$$
la parte antisimmetrica di $ \Nabla\mathbf{u}$.
\`E immediato verificare che
\begin{equation}
\mathbf{S}=\frac{1}{2}{\setlength\arraycolsep{2pt} 
\left(\begin{array}{ccc} 
0&  -\xi_3 &\xi_2\\ 
\xi_2&  0 &-\xi_1\\ 
-\xi_2&  \xi_1 &0\\
\end{array}\right) 
}
\end{equation}
e quindi
$$\mathbf{S} \cdot \mathbf{h} = \frac{1}{2} \boldsymbol{\xi} \times \mathbf{h}$$.
\qed%è bruttissimo XD
\end{proof}

La matrice $\mathbf{D}$ è detta tensore della deformazione. Dato che $\mathbf{D}$ simmetrica esiste una base ortonormale in cui è diagonale
\begin{equation}
\mathbf{D}={\setlength\arraycolsep{2pt} 
\left(\begin{array}{ccc} 
d_1 & 0 & 0 \\ 
0 & d_2 & 0 \\ 
0 & 0 & d_3 \\
\end{array}\right) 
}
\end{equation}.
Considerando solo il contributo di questo termine l'equazione \ref{eq:helmoltz} diventa 
$$\frac{d h_i}{dt} = d_i h_i$$
Si tratta quindi di una contrazione o espansione nelle direzione della base ortonormale trovata.

Inoltre è possibile calcolare la variazione di volume
\begin{equation}
\begin{split}
\frac{d}{dt}h_1 h_2 h_3 &= \left [ \frac{dh_1}{dt} \right ] h_2h_3 + \left [ \frac{dh_2}{dt} \right ] h_1h_3 + \left [ \frac{dh_3}{dt} \right ] h_1h_2 \\
                                        &= \traccia \mathbf{D} h_1 h_2 h_3  \\
                                        &= \traccia  \frac{1}{2} [ \Nabla\mathbf{u} + \Nabla\mathbf{u} ^T] h_1 h_2 h_3 \\ 
                                        &= \Nabla \cdot \mathbf{u} h_1 h_2 h_3
\end{split}
\end{equation}
perché la traccia è invariante rispetto a cambi di coordinate. Si riottiene quindi che la deformazione è proporzionale a $ \Nabla \cdot \mathbf{u}$.

Il termine $\mathbf{u}(\mathbf{x})$ è una traslazione.

Il termine $\frac{1}{2}\boldsymbol{\xi}(\mathbf{x})\times\mathbf{h}$, usando lo stesso ragionamento usato nel caso di $\mathbf{D}$
$$\frac{d\mathbf{h}}{dt}=\frac{1}{2}\boldsymbol{\xi}(\mathbf{x})\times\mathbf{h}$$
la cui soluzione è
$$\mathbf{h}(t)=\mathbf{R}(t,\boldsymbol{\xi}(\mathbf{x}))\mathbf{h}(0)$$
con $\mathbf{R}$ rotazione rigida di angolo $t$ e asse $\boldsymbol{\xi}$. 

Le ipotesi fatte fino ad ora impediscono il presentarsi di qualsiasi forza tangenziale e quindi di qualsiasi meccanismo che possa causare o fermare una rotazione del fluido, cioè in ogni movimento del fluido la rotazione viene conservata. Data la stretta connessione tra rotazione e vorticità ci si aspetta che la vorticità sia connessa a questo fatto. 

[...][circolazioni,teorema di helmotz, teorema di kelvin]

\subsection{L'equazione di Navier Stokes}
Nei fluidi reali la quantità di moto viene diffusa anche grazie ai movimenti molecolari.
Rinunciamo ora all'ipotesi che il fluido sia ideale, e quindi ammettiamo anche forze di stress non in direzione normale.
Il teorema di Cauchy ci garantisce che la forza che agisce su $S$ è una 
$$F \, dA = -p(\mathbf{x},t) \mathbf{n} + \boldsymbol{\sigma}(\mathbf{x},t) \cdot \mathbf{n}$$
dove $\boldsymbol{\sigma}$ è il tensore degli stress.
Con questa nuova forza la legge di conservazione del momento diventa
$$\frac{d}{dt} \int_{\mathcal{W}_t} \rho \mathbf{u} \, dV = \int_{\partial \mathcal{W}_t} (p \mathbf{n} - \boldsymbol{\sigma} \cdot \mathbf{n} )\, dA$$
Faremo inoltre le seguenti ipotesi su $\boldsymbol{\sigma}$: \begin{itemize}
  \item $\boldsymbol{\sigma}$ dipende linearmente dal gradiente della velocità $\Nabla \mathbf{u}$;
  \item $\boldsymbol{\sigma}$ è invariante rispetto a rotazioni rigide, cioè data $\mathbf{U}$ matrice ortogonale
$$\boldsymbol{\sigma}(\mathbf{U} \cdot \Nabla \mathbf{u} \cdot \mathbf{U}^{-1}) = \mathbf{U} \cdot \boldsymbol{\sigma}( \Nabla \mathbf{u} ) \cdot \mathbf{U}^{-1}$$
\item $\boldsymbol{\sigma}$ è simmetrico.
\end{itemize} 

Dato che $\boldsymbol{\sigma}$ è simmetrico dipenderà solo dalla parte simmetrica di $\Nabla \mathbf{u}$, cioè dalla deformazione $\mathbf{D}$. Inoltre $\boldsymbol{\sigma}$ e $\mathbf{D}$ possono essere diagonalizzate simultaneamente in quanto commutano essendo $\boldsymbol{\sigma}$ una funzione lineare di $\mathbf{D}$.
Quindi gli autovalori di sigma sono funzioni lineari e simmetriche di quelli di D.
$$\sigma_i = \lambda (d_1 + d_2 + d_3) +2 \mu d_i$$
e quindi 
$$\boldsymbol{\sigma} = \lambda ( \Nabla\cdot \mathbf{u}) \mathbf{I} + 2 \mu \mathbf{D}
                                     = 2 \mu [ \mathbf{D} - \frac{1}{3} ( \Nabla\cdot \mathbf{u}) \mathbf{I} ] + \zeta ( \Nabla\cdot \mathbf{u})  \mathbf{I}
$$
$\mu$ è il primo coefficiente di viscosità e $\zeta$ è il secondo coefficiente di viscosità.

Usando il teorema del trasporto ora si ottiene l'equazione di Navier-Stokes

$$\rho \frac{D\mathbf{u}}{Dt} = -\nabla p + (\lambda+\mu) \nabla(\Nabla \cdot \mathbf{u}) + \mu \boldsymbol{\Delta} \mathbf{u}$$

Nel caso di fluidi incomprimibili le equazioni diventano
\begin{equation}
\boxed{
\begin{aligned}
\frac{D\mathbf{u}}{Dt} &= - \nabla p + \nu \boldsymbol{\Delta} \mathbf{u} \\
\Nabla \cdot  \mathbf{u} &= 0 \\
\mathbf{u} &= 0 \quad \text{su} \quad \partial \mathcal{D}
\end{aligned}
}
\end{equation}
L'ultima equazione è la cosiddetta condizione \emph{no-slip}, per cui il fluido è a riposo sulle pareti.

Nel caso dei fluidi incomprimibili quindi il tensore di stress $\boldsymbol{\sigma}$ dipende dal moltiplicatore di Lagrange relativo al vincolo di incomprimibilità $p$, anche detto pressione idrostatica, e dalla parte simmetrica di $\Nabla\mathbf{u}$, $\mathbf{D}=\frac{1}{2}(\Nabla\mathbf{u}+(\Nabla\mathbf{u})^{T})$
$$\boldsymbol{\sigma}=-p\mathbf{I}+\eta(\Nabla\mathbf{u}+(\Nabla\mathbf{u})^{T})$$
la cui divergenza è
$$\Nabla\cdot\boldsymbol{\sigma}=-\nabla p+\eta\Delta\mathbf{u}$$.
[numero di Reynolds][metodi di proiezione per la pressione e equazione di Stokes]

\subsection{Le equazioni della fluidodinamica in forma debole}
%The Finite Element Immersed Boundary Method, L. Heltai ,Ph.D. Thesis , 2006, capitolo 1 19-35
Scriviamo ora l'equazione di Navier-Stokes in forma variazionale.

Sia
$$\mathbf{V} = H^1_0(\Omega)^d = \bigg \{ \mathbf{v} \in L^2(\Omega)^d \quad \tc \quad \Nabla \mathbf{v} \in L^2(\Omega)^{d \times d} \quad \text{e} \quad \mathbf{v}\mid_{\partial \Omega} = 0 \bigg \} $$
lo spazio delle velocità e 
$$Q = L^2_0(\Omega) = \left  \{ q \in L^2(\Omega) \quad \tc \quad \int_{\Omega}q=0 \right  \}$$ lo spazio delle pressioni. Indichiamo con $V'$ lo spazio duale di $V$.

Dati $b \in \mathbf{V}'$ e $\mathbf{u}_0 \in \mathbf{V}$, per ogni $t \in ]0,T[$ trovare $(\mathbf{u},p) \in V \times Q$ tale che
\begin{multline*}
\rho \left ( \frac{d}{dt}\ps{\mathbf{u}(t)}{\mathbf{v}} + \ps{\mathbf{u}(t)\cdot \Nabla\mathbf{u}(t)}{\mathbf{v})}\right)+ \nu\ps{\Nabla\mathbf{u}(t)}{\Nabla\mathbf{v}}-\ps{\Nabla \mathbf{v}}{p(t)}=\ps{\mathbf{b}}{\mathbf{v}} \qquad \forall \mathbf{v}\in\mathbf{V}
\end{multline*}
$$\ps{\Nabla\cdot\mathbf{u}(t)}{q}=0 \qquad \forall q\in\mathbf{Q}$$
$$\mathbf{u}(\mathbf{x},0)=\mathbf{u}_0(\mathbf{x}) \qquad \forall \mathbf{x}\in\Omega$$

Definendo i seguenti operatori associati ai termini dell'equazione di Navier-Stokes
\begin{equation}
\begin{aligned}
(M\mathbf{u})(\mathbf{v})&=\rho\ps{\mathbf{u}}{\mathbf{v}} \quad &\mathbf{u},\mathbf{v}\in \mathbf{V}\\
(A\mathbf{u})(\mathbf{v})&=a(\mathbf{u},\mathbf{v})=\eta\ps{\Nabla\mathbf{u}}{\Nabla\mathbf{v}} \quad &\mathbf{u},\mathbf{v}\in \mathbf{V}\\
(N(\mathbf{w})\mathbf{u})(\mathbf{v})&=c(\mathbf{w},\mathbf{u},\mathbf{v})=\rho\ps{\mathbf{w}\cdot\Nabla\mathbf{u}}{\Nabla\mathbf{v}} \quad &\mathbf{w},\mathbf{u},\mathbf{v}\in \mathbf{V}\\
(B\mathbf{u})(q)&=b(\mathbf{u},q)=\ps{\Nabla\cdot\mathbf{u}}{q} \quad &\mathbf{u}\in \mathbf{V},q\in Q
\end{aligned}
\end{equation}
possiamo riscrivere le equazioni come
\begin{equation}
\begin{aligned}
&\rho\frac{d}{dt}\ps{\mathbf{u}}{\mathbf{v}}+a(\mathbf{u},\mathbf{v})+c(\mathbf{w},\mathbf{u},\mathbf{v})+b(\mathbf{v},p)=\ps{\mathbf{b}}{\mathbf{u}} \quad &\forall\mathbf{v}\in\mathbf{V}\\
&b(\mathbf{u},q)=0 \quad &\forall q\in Q\\
&\mathbf{u}(\mathbf{x},0)=\mathbf{u}_0(\mathbf{x})&
\end{aligned}
\end{equation}
oppure in termini di operatori degli spazi duali
\begin{equation}
\begin{aligned}
&M\frac{\partial\mathbf{u}}{\partial t}+A\mathbf{u}+N(\mathbf{u})\mathbf{u}+B^{T}p=\mathbf{b}\\
&B\mathbf{u}=0\\
&\mathbf{u}(\mathbf{x},0)=\mathbf{u}_0(\mathbf{x})&
\end{aligned}
\end{equation}
[vedi 31,32,33,34]

\paragraph{Stima dell'energia}
\begin{theorem}
Per ogni $t \in ]0,T[$ sia $(\mathbf{u}(t),p(t)) \in V \times Q$ la soluzione del problema con $\mathbf{b}=0$, allora
$$\rho \frac{1}{2} \frac{d}{dt} \Vert \mathbf{u} (t) \Vert^2_{0,\Omega} + \eta \Vert \Nabla \mathbf{u} (t) \Vert^2_{0,\Omega} = 0$$
\end{theorem}
\begin{proof}
Per dimostrarlo si prendano $\mathbf{v}=\mathbf{u}$ e $q=p$ come funzioni di test nelle equazioni. Allora $b(\mathbf{u},p)=0$ e $c(\mathbf{u},\mathbf{u},\mathbf{u})=0$.
$$
\rho\ps{\frac{\partial \mathbf{u}}{\partial t}}{\mathbf{u}}+\eta\ps{\Nabla\mathbf{u}}{\Nabla\mathbf{u}} = 
\rho\frac{1}{2}\frac{\partial} {\partial t}\Vert\mathbf{u}\Vert^2+\eta\Vert \Nabla\mathbf{u}\Vert^2 = 0
$$
\end{proof}
Il teorema afferma che la variazione di energia cinetica è pari all'energia dissipata dai fenomeni viscosi.
I fluidi governati dalle equazioni di Navier-Stokes incomprimibili non possono accumulare energia se non sottoforma di energia cinetica dato che il fluido è incomprimibile e lo stress $\boldsymbol{\sigma}$ non dipende dalla deformazione del fluido. 

\section{Teoria matematica dell'elasticità}
%Mersden Mathematical Foundation of Elasticity
Sia $\mathcal{B} \in \mathbb{R}^3$ la chiusura di un aperto con bordo a tratti liscio. $\mathcal{B}$ è detta configurazione di riferimento del corpo in esame.
Una configurazione o deformazione di $\mathcal{B}$ è una mappa $\phi: \mathcal{B} \to \mathbb{R}^3$ sufficientemente liscia, che preserva l'orientamento e invertibile. I punti $\mathbf{X}=(X_1,X_2,X_3) \in \mathcal{B}$ sono detti punti materiali, mentre i punti $\mathbf{x}=(x_1,x_2,x_3) \in \mathbb{R}^3$ sono detti punti spaziali. 
Un moto di $\mathcal{B}$ è una famiglia di configurazioni dipendente dal tempo $\phi: \mathcal{B} \times [-T,T] \to R^3$.
La velocità del punto materiale $\mathbf{X}$ è data da 
$$\mathbf{V}(\mathbf{X},t) = \frac{\partial \phi}{\partial t}(\mathbf{X},t)$$,
mentre la velocità del punto spaziale $\mathbf{x}=\phi(\mathbf{X})$ è data da
$$\mathbf{v}(\mathbf{x},t) = \mathbf{V}(\mathbf{X},t)$$.
L'accelerazione è data da
$$\mathbf{A}(\mathbf{X},t) = \frac{\partial^2 \phi}{\partial t^2} = \frac{\partial \mathbf{V}}{\partial t} = \frac{\partial \mathbf{v}}{\partial t}+\mathbf{v}\cdot\Nabla\mathbf{v}$$.
Più in generale [...].
\paragraph{Gradiente di deformazione}
La matrice delle derivate parziali di $\boldsymbol{\phi}$ 
$$ \mathbf{F} = D\boldsymbol{\phi}$$
è detta gradiente di deformazione.
$\boldsymbol{\phi}$ preserva l'orientazione quindi $\mathbf{F} > 0$. 

$\mathbf{F}$ può essere decomposto in maniera univoca come 
$$\mathbf{F}=\mathbf{R}\mathbf{U}=\mathbf{V}\mathbf{R}$$
dove $\mathbf{R}$ è una matrice ortogonale, e quindi rappresenta una rotazione, mentre $\mathbf{U}$ e $\mathbf{V}$ sono matrici simmetriche definite positive e sono dette rispettivamente tensore di stress destro e sinistro. 
Inoltre $\mathbf{U}=\sqrt{\mathbf{F}^T\mathbf{F}}$ e $\mathbf{V}=\sqrt{\mathbf{F}\mathbf{F}^T}$.
Siano $\mathbf{C}=\mathbf{F}^T\mathbf{F}=\mathbf{U}^2$ il tensore destro di Cauchy-Green e $\mathbf{b}=\mathbf{F}\mathbf{F}^T=\mathbf{V}^2$ il tensore sinistro di Cauchy-Green.
Dato che $\mathbf{U}$ e $\mathbf{V}$ sono simili, simmetriche e definite positive allora hanno gli stessi autovalori $\lambda_1$, $\lambda_2$ e $\lambda_3$ reali e positivi, detti tensioni principali. I rispettivi autovettori sono detti direzioni principali. Il valore di $\lambda_i$ rappresenta lo stiramento nella direzione del rispettivo autovettore.
\paragraph{Leggi di Conservazione}
\subparagraph{Conservazione della Massa}
Sia $\boldsymbol{\phi}(\mathbf{x},t)$ un moto di $\mathcal{B}$ e $\rho(\mathbf{x},t)$ la densità di massa del corpo deformato e $\rho_{\mathcal{B}}(\mathbf{X})$ la densità di massa del corpo non deformato. 
Sia $\mathcal{U}\in\mathcal{B}$ una porzione del corpo $\mathcal{B}$ e  $\mathcal{U}_t=\boldsymbol{\phi}(\mathcal{U},t)$ la stessa porzione al tempo $t$. Allora 
$$\int_{\mathcal{U}}\rho_{\mathcal{B}}(\mathbf{X})\, d\mathbf{X} = \int_{\mathcal{U}_t}\rho(\mathbf{x},t)\, d\mathbf{x} = $$
[...] e quindi
$$\int_{\mathcal{U}_t}\frac{\partial\rho}{\partial t}(\mathbf{x},t)\, d\mathbf{x} =  -\int_{\partial\mathcal{U}_t} \mathbf{J} \cdot \mathbf{n} \, da$$
\subparagraph{Conservazione del Momento}
Sia $\mathbf{b}(\mathbf{x},t)$ la forza applicata al corpo $\mathcal{B}$ per unità di massa, e sia $\boldsymbol{\tau}(\mathbf{x},t)$ la forza di superficie applicata per unità di superficie. La coppia $(\mathbf{b},\boldsymbol{\tau})$ è detto carico applicato al corpo. Sia $\mathbf{t}(\mathbf{x},t,\mathbf{n})$ la forza di stress interna al corpo per unità di area nella direzione $\mathbf{n}$. La seconda legge di Newton diventa quindi
$$\frac{d}{dt}\int_{\mathcal{U}_t}\rho\mathbf{v}\, d\mathbf{x} = \int_{\partial \mathcal{U}_t}\mathbf{t}\, da + \int_{\mathcal{U}_t}\rho\mathbf{b}\, d\mathbf{x} $$.
Per il teoreme di Cauchy se vale la legge di conservazione del momento allora $\mathbf{t}$ dipende linearmente da  $\mathbf{n}$, e quindi esiste un tensore $\boldsymbol{\sigma}$ tale che 
$$\mathbf{t}(\mathbf{x},t,\mathbf{n})= \boldsymbol{\sigma}(\mathbf{x},t)\mathbf{n}$$. Questo tensore è detto tensore degli stress di Cauchy. Sfruttando il teorema della divergenza di ottiene
$$\frac{d}{dt}\int_{\mathcal{U}_t}\rho\mathbf{v}\, d\mathbf{x} = \int_{\mathcal{U}_t}\Nabla\cdot\boldsymbol{\sigma}+\rho\mathbf{b}\, d\mathbf{x} $$.
Data l'arbitrarietà di ${\mathcal{U}_t}$ si ottiene l'equazione del moto di Cauchy
$$\rho\frac{d\mathbf{v}}{dt}=\Nabla\cdot\boldsymbol{\sigma}+\rho\mathbf{b}$$. 
\subparagraph{In coordinate lagrangiane}
\'E possibile esprimere il tensore di Cauchy in coordinate lagrangiane sfruttando le quantità materiali al posto di quelle spaziali.
Il primo vettore di stress di Piola-Kirchhoff $\mathbf{T}(\mathbf{X},t,\mathbf{n})$ è un vettore parallelo al vettore di stress di Cauchy $\mathbf{t}(\mathbf{x},t,\mathbf{n})$, ma misura la forza per unità di area non deformata.
[...]
Sia $\mathbf{P}=J\boldsymbol{\sigma}\mathbf{F}^{-T}$ il primo tensore di Piola-Kirchhoff allora
$$\frac{d}{dt}\int_{\mathcal{U}}\rho_{\mathcal{B}}\mathbf{V}\, d\mathbf{X} = \int_\mathcal{U}\Nabla\cdot\mathbf{P}+\rho_{\mathcal{B}}\mathbf{B}\, d\mathbf{X} $$.
Sia $\mathbf{S}=\mathbf{F}^{-1}\mathbf{P}$ il secondo tensore di Piola-Kirchhoff.

\paragraph{Materiali elastici}
Per risolvere le equazioni del moto bisogna determinare come la forza nella seconda equazione di Newton dipende dalla posizione e dalla velocità e come lo stress dipende dal moto nell'equazione del moto di Cauchy. 
Un materiale è elastico se esiste una funzione detta funzione costitutiva $\hat{\mathbf{P}}$ tale che
$$\mathbf{P}(\mathbf{X},t)=\hat{\mathbf{P}}(\mathbf{X},\mathbf{F}(\mathbf{X},t)) $$
cioè il primo tensore di Piola-Kirchhoff dipende dalla deformazione ma non dalle sue derivate, ad esempio dalla velocità come nel caso dei fluidi e dei corpi viscosi.
\paragraph{Materiali iperelastici}
Un materiale si dice iperelastico se esiste una funzione detta funzione energia  $\widehat{\mathbf{W}}$ tale che 
$$\widehat{\mathbf{P}}(\mathbf{X},\mathbf{F}(\mathbf{X},t))=\rho_{\mathcal{B}}\frac{\partial \widehat{\mathbf{W}}}{\partial \mathbf{F}} $$
\paragraph{\emph{Indipendenza dal sistema di riferimento}}
Le leggi costitutive del corpo in esame non devono dipendere dal particolare sistema di riferimento usato per descriverlo.
In particolare sia $\mathbf{Q}$ una matrice ortogonale, allora 
$$\mathbf{T}(\mathbf{X},\mathbf{F},\mathbf{N}) = \widehat{\mathbf{P}}(\mathbf{X},\mathbf{F})\mathbf{N} $$
$$\widehat{\mathbf{P}}(\mathbf{X},\mathbf{Q}\mathbf{F}) = \mathbf{Q}\widehat{\mathbf{P}}(\mathbf{X},\mathbf{F}) $$
Per un materiale iperelastico 
$$\widehat{\mathbf{W}}(\mathbf{X},\mathbf{Q}\mathbf{F}) = \widehat{\mathbf{W}}(\mathbf{X},\mathbf{F}) $$
e quindi
$$\widehat{\mathbf{W}}(\mathbf{X},\mathbf{F}) = \widehat{\mathbf{W}}(\mathbf{X},\sqrt{\mathbf{C}}) $$
L'indipendenza dal sistema di riferimento implica la conservazione momento angolare e la simmetria dei tensori di stress.
\paragraph{Materiali isotropici}
Un materiale è omogeneo se $\widehat{\mathbf{P}}$ non dipende esplicitamente da $\mathbf{X}$, cioè il comportamento del materiale non dipende dal punto del materiale in esame ma solo dalla sua deformazione.

Un materiale è isotropico se
$$\widehat{\mathbf{P}}(\mathbf{X},\mathbf{Q}\mathbf{F}) = \widehat{\mathbf{P}}(\mathbf{X},\mathbf{F}) $$
per ogni matrice ortogonale $\mathbf{Q}$.
Inoltre se un materiale è iperelastico vale 
$$\widehat{\mathbf{W}}(\mathbf{X},\mathbf{F}\mathbf{Q}) = \widehat{\mathbf{W}}(\mathbf{X},\mathbf{F}) $$

Un materiale è iperelastico, indipendente dal sistema di riferimento, omogeneo e isotropico se e solo se
$$\widehat{\mathbf{W}}(\mathbf{F}) = \Phi(\lambda_1,\lambda_2,\lambda_3) $$
con $\Phi$ funzione simmetrica degli sforzi principali $\lambda_1$, $\lambda_2$ e $\lambda_3$.
Si può mostrare inoltre che
$$\mathbf{S}=\alpha_0\mathbf{I}+\alpha_1\mathbf{C}+\alpha_2\mathbf{C}^2 $$
e
$$\boldsymbol{\sigma}=\beta_0\mathbf{I}+\beta_1\mathbf{b}+\beta_2\mathbf{b}^2 $$
con $\alpha_i$ e $\beta_i$ funzioni scalari degli invarianti di $\mathbf{C}$  e di $\mathbf{b}$.
\paragraph{Elasticità lineare}
Quando il corpo in esame è soggetto solo a piccoli spostamenti e a piccole deformazioni è possibile ottenere una teoria semplificata linearizzando le equazioni non lineari del moto.

Sia $\widehat{\mathbf{P}}$ la funzione costitutiva di un materiale elastico omogeneo non lineare tale che $\widehat{\mathbf{P}}(\mathbf{I})=0$, cioè la configurazione di riferimento ha stress nullo.
Sia $\phi_\epsilon(\mathbf{X},t)$ una famiglia di moti dipendenti dal parametro $\epsilon \ll1$ tale che $\phi_0(\mathbf{X},t) = \mathbf{X}$. Quindi $\phi_0$ soddisfa le equazioni del moto
$$\rho_\mathcal{B}\frac{\partial \mathbf{V}}{\partial t} = \Nabla\cdot\widehat{\mathbf{P}}(\mathbf{F}) $$.
Si supponga che anche $\phi_\epsilon$ soddisfi la stessa equazione per ogni $\epsilon$.
Sia 
$$\phi_\epsilon(\mathbf{X},t) = \mathbf{X}+\epsilon\mathbf{u}(\mathbf{X},t) + \mathcal{O}(\epsilon^2)$$
l'espansione in serie di $\phi_\epsilon$.
Il problema di riduce a trovare $\mathbf{u}$.
Osservato che 
$$\mathbf{u}=\frac{\partial}{\partial \epsilon}\phi_\epsilon \big|_{\epsilon=0}$$
e
$$\frac{\partial}{\partial t}\mathbf{u}=\frac{\partial}{\partial \epsilon}\mathbf{V}_\epsilon \big|_{\epsilon=0}$$
differenziando l'equazione (?.?) si ottiene
$$\rho_{\mathcal{B}} \frac{\partial}{\partial \epsilon}\mathbf{V}_\epsilon = \Nabla\cdot\widehat{\mathbf{P}}(\mathbf{F_\epsilon)}$$
che in $\epsilon=0$ vale
$$\rho_{\mathcal{B}} \frac{\partial^2\mathbf{u}^i}{\partial t^2} = \frac{\partial}{\partial \mathbf{X}^j}(\frac{\partial\widehat{\mathbf{P}}^{ij}}{\partial\mathbf{F}^k_l}(\mathbf{I})\frac{\partial\mathbf{u}^k}{\partial\mathbf{X}^l})$$
Definiamo il tensore classico di elasticità come
$$ c_{ijkl}(\mathbf{X})= \frac{\partial\widehat{\mathbf{P}}^{ij}}{\partial\mathbf{F}^k_l}(\mathbf{I})$$.
Se il materiale è omogeneo allora i valori di $c_{ijkl}$ sono costanti in tutto il corpo e l'equazione dell'elasticità lineare diventa
$$\rho_{\mathcal{B}} \frac{\partial^2\mathbf{u}^i}{\partial t^2} = \frac{\partial^2\mathbf{u}^k}{\partial\mathbf{X}^j\partial\mathbf{X}^l}$$.
Se il materiale è omogeneo e isotropico allora esistono due costani $\lambda$ e $\mu$ dette moduli di Lamè per cui
$$c_{ijkl}=\lambda\delta_{ij}\delta_{kl}+\mu(\delta_{ik}\delta_{jl}+\delta_{il}\delta_{jk})$$.
\paragraph{Un fluido elastico}
Si consideri un corpo non lineare iperelastico per cui
$$\widehat{W}(\mathbf{F}) = h(|\mathbf{F}|)$$.
Allora è omogeneo, indipendente dal sistema di riferimento e isotropico e il primo tensore di Piola-Kirchhoff vale
$$\widehat{\mathbf{P}}=\rho_{\mathcal{B}}\frac{\partial \widehat{W}}{\partial \mathbf{F}} = \rho_{\mathcal{B}}h'(J)\frac{\partial \mathbf{J}}{\partial \mathbf{F}}$$
con $J = |\mathbf{F}|$.
Osservato che 
$$\frac{\partial J}{\partial \mathbf{F}} = J\mathbf{F}^{-T}$$
allora il tensore di stress di Cauchy vale
$$\boldsymbol{\sigma}=(\frac{1}{J})\widehat{\mathbf{P}}\mathbf{F}^{T}=\rho_{\mathcal{B}}h'(J)\mathbf{I}$$.
Dato che
$$p(\rho)=-\rho_{\mathcal{B}}h'(J)$$
allora
$$\boldsymbol{\sigma}=-p(\rho)\mathbf{I}$$.
L'equazione del moto di Cauchy diventa
$$\rho\frac{D \mathbf{u}}{Dt} = -\Nabla p+\rho\mathbf{b}$$
che è l'equazione di Eulero per i fluidi perfetti comprimibili, che quindi sono un sottoinsieme dei materiali elastici.
Nei fluidi viscosi invece $\boldsymbol{\sigma}$ dipende non solo da $\mathbf{F}$ ma anche da $\Nabla\mathbf{v}$.
[distinguere fluidi da solidi mediante gruppo di simmetria]
\paragraph{Incomprimibilità}
Un materiale è incomprimibile se...
e quindi $J\equiv 1$ o $\Nabla\cdot\mathbf{v}=0$.
Il moltiplicatore di Lagrange associato a questo vincolo
$$\mathbf{P}(\mathbf{X},t)=-p\mathbf{F}^{-T}+\widehat{\mathbf{P}}(\mathbf{X},\mathbf{F}(\mathbf{X},t))$$.

Per $\widehat{\mathbf{P}}=0$  si ottiene l'equazione di Eulero per i fluidi perfetti incomprimibili
$$\rho\frac{D \mathbf{v}}{Dt} = -\Nabla p+\rho\mathbf{b}$$
$$\Nabla\cdot\mathbf{v}=0$$
\paragraph{Materiali di Mooney-Rivlin}

%Heltai

\section{Teoria dell'elasticità strikes back}
[coordinate euleriane e lagrangiane]

\paragraph{Il teorema del trasporto}

\begin{theorem}[del Trasporto di Reynolds]
	Sia $\Phi(\mathbf{x},t)$ una funzione scalare o vettoriale, $\mathcal{P}\subset\mathcal{B}$ una porzione regolare del corpo in esame e  $\mathcal{P}_t = \mathbf{X}(\mathcal{P},t)$ la stessa porzione al tempo $t$ allora
	$$\forall t \in (0,T)\> \frac{d}{dt}\int_{\mathcal{P}_t}\Phi(\mathbf{x},t)\,d\mathbf{x}=\int_{\mathcal{P}_t}\frac{D\Phi}{Dt}(\mathbf{x},t)+\Phi(\mathbf{x},t)\Nabla\cdot\mathbf{u}(\mathbf{x},t)\,d\mathbf{x}$$
\end{theorem}

Le equazioni di conservazione possono essere ricavate applicando il teorema del trasporto.

\paragraph{Conservazione del volume}
Applicando il teorema di Reynolds alla funzione $\Phi \equiv 1$ si ottiene la legge di conservazione del volume
$$\frac{d}{dt}\int_{\mathcal{P}_t}\,d\mathbf{x}=\int_{\mathcal{P}_t}\Nabla\cdot\mathbf{u}\,d\mathbf{x}=\int_{\partial\mathcal{P}_t}\mathbf{u}\cdot\mathbf{n}\,d\mathbf{x} $$.
Se il materiale è incomprimibile allora $\Nabla\cdot\mathbf{u}=0$.

\paragraph{Conservazione della massa}
Sia $M(\mathbf{s})$ la densità di massa del punto materiale $\mathbf{s}$ e 
$$\rho(\mathbf{x},t)=M(\mathbf{X}(\mathbf{x},t))$$
la densità di massa in coordinate euleriane.
Applicando il principio di conservazione della massa si ottiene
$$\frac{d}{dt}\int_{\mathcal{P}_t}\rho\,d\mathbf{x}=\int_{\mathcal{P}_t}\frac{D\rho}{Dt}+\rho\Nabla\cdot\mathbf{u}\,d\mathbf{x}=0$$.
Se inoltre il materiale è incomprimibile allora 
$$\frac{D\rho}{Dt}(\mathbf{x},t)=0 $$.
e applicando il teorema di Reynold alla funzione $\rho\Phi$ si ottiene
$$\frac{d}{dt}\int_{\mathcal{P}_t}\rho\Phi\,d\mathbf{x}=\int_{\mathcal{P}_t}\frac{D(\rho\Phi)}{Dt}+\rho\Phi\Nabla\cdot\mathbf{u}\,d\mathbf{x}=\int_{\mathcal{P}_t}(\frac{D\rho}{Dt}+\rho\Nabla\cdot\mathbf{u})\Phi+\rho\frac{D\Phi}{Dt}\,d\mathbf{x}=\int_{\mathcal{P}_t}\rho\frac{D\Phi}{Dt}\,d\mathbf{x}$$.
Quindi in un materiale incomprimibile la densità di massa si comporta come una costante per quanto riguarda l'operazione di derivazione materiale.

\paragraph{Conservazione della quantità di moto}
Sia
$$\mathbf{p}(\mathcal{P},t)=\int_{\mathcal{P}_t}\rho\mathbf{u}\,d\mathbf{x}$$
la quantità di moto della porzione di corpo $\mathcal{P}$
e
$$\mathbf{L}(\mathcal{P},t)=\int_{\mathcal{P}_t}\rho\mathbf{x}\times\mathbf{u}\,d\mathbf{x}$$
il momento angolare.
Applicando il teorema di Reynolds si ottiene
$$\frac{D\mathbf{p}}{Dt}=\int_{\mathcal{P}_t}\rho\frac{D\mathbf{u}}{Dt}\,d\mathbf{x}$$
$$\frac{D\mathbf{L}}{Dt}=\int_{\mathcal{P}_t}\rho\mathbf{x}\times\frac{D\mathbf{u}}{Dt}\,d\mathbf{x}$$
Enunciamo ora la prima legge di Newton
\begin{theorem}
	Sia $\mathbf{f}(\mathcal{P},t)$ la risultante delle forze applicate alla porzione di corpo $\mathcal{P}$ al tempo $t$ allora
	$$\mathbf{f}((\mathcal{P},t))=\frac{D\mathbf{p}}{Dt}((\mathcal{P},t))$$
\end{theorem}
Si possono distinguere due tipi di forze, le forze di volume che sono applicate ad ogni punto del materiale, e le forze di contatto, responsabili dell'interazione tra sottoinsiemi del materiale.
Siano $\mathbf{b}(\mathbf{x},t)$ la densità di forze di volume e $\mathbf{t}(\mathbf{x},t,\mathbf{n})$ la densità di forze di contatto. Verrano considerati solo materiali di Cauchy in cui la densità di forza di contatto dipende solo dalla normale alla superficie e non dalla curvatura della stessa.
La forza applicata da una regione del corpo su di un'altra regione attraverso la superficie $\mathcal{S}$ è data da
$$\int_{\mathcal{S}}\mathbf{t}(\mathbf{n},\mathbf{x},t)\,da$$.
La prima legge di Newton per un materiale di Cauchy diventa
$$\mathbf{f}((\mathcal{P},t))=\frac{D\mathbf{p}}{Dt}((\mathcal{P},t)) = \int_{\partial\mathcal{P}_t}\mathbf{t}+\int_{\mathcal{P}_t}\mathbf{b}$$
Che è equivalente a chiedere che valga il principio dei lavori virtulali, cioè che per ogni $\mathcal{P}$, $t$ e per ogni $\mathbf{n}$ esista un tensore simmetrico $\boldsymbol{\sigma}$ detto tensore di stress di Cauchy tale che $\mathbf{t}(\mathbf{x},t,mathbf{n})=\boldsymbol{\sigma}(\mathbf{x},t)\mathbf{n}$ e
$$\int_{\mathcal{P}_t}\frac{D\mathbf{u}}{Dt}\cdot\mathbf{v}\,d\mathbf{x}+\int_{\mathcal{P}_t}\boldsymbol{\sigma}\boldsymbol{:}\Nabla\mathbf{v}\,d\mathbf{x}=\int_{\partial\mathcal{P}_t}\boldsymbol{\sigma}\mathbf{n}\cdot\mathbf{v}\,d\mathbf{x}+\int_{\mathcal{P}_t}\mathbf{b}\cdot\mathbf{v}\,d\mathbf{x}$$
per ogni funzione di test $\mathbf{v}$ sufficientemente regolare a supporto compatto in $\Omega$.
[$$\boldsymbol{\sigma}\boldsymbol{:}\Nabla\mathbf{v}=\sigma_{\alpha\beta}\frac{\partial v_{\alpha}}{\partial x_{\beta}}$$].
Se $\boldsymbol{\sigma}$ è sufficiente regolare allora vale la versione puntuale dell'equazione
$$\rho\frac{D\mathbf{u}}{Dt}=\Nabla\cdot\boldsymbol{\sigma}+\mathbf{b}$$.
Si può notare che in dimensione $d=2$ l'equazione fornisce $2$ condizioni per $2(\mathbf{u})+3(\boldsymbol{\sigma})$ incognite, mentre in dimensione $d=3$ l'equazione fornisce $3$ condizioni per $3(\mathbf{u})+6(\boldsymbol{\sigma})$.
I rimanenti gradi di libertà possono essere utilizzati per modellizzare vari tipi di materiali differenti, ad esempio i fluidi viscosi e i materiali elastici.


Un corpo si dice elastico se tende a ritornare nella sua configurazione originale ( o di equilibrio ) dopo essere stato deformato. Le particelle di un corpo elastico tendono a recuperare la loro posizione relativa rispetto alle altre particelle, mentre in un fluido il comportamento è stabilito solo dalla loro velocità relativa.

\paragraph{Materiali iperelastici}

Un materiale incomprimibile si dice iperelastico se è possibile definire una funzione di densità di energia potenziale

$W( \mathbb{F},\mathbf{s}) > 0$

associata alla deformazione $\mathbb{F}$.

Il comportamento di un materiale iperelastico dipende solo dalla sua configurazione attuale e non presenta fenomeni di isteresi.

Sia $E [  \mathbf{X}(t) ] = \int_{\omega} W( \mathbb{F}, \mathbf{s} )) \, d\mathbf{s}$ l'energia potenziale totale del corpo.

Grazie ad alcune considerazione fisiche possiamo caratterizzare la funzione $W$:

1. In assenza di deformazione non c'è energia accumulata nel corpo $W(\mathbb{I}) = 0$

2. la densità di energia diverge a $+\infty$ quando il corpo viene compresso o espanso indefinitamente  

$\vert \mathbb{F} \vert \rightarrow 0 \Rightarrow W(\mathbb{F}) \rightarrow +\infty$

$\vert \mathbb{F} \vert \rightarrow +\infty \Rightarrow W(\mathbb{F}) \rightarrow +\infty$

3.  la densità di energia non varia in seguito a moti rigidi del corpo, cioè se $\mathbb{F}^*$ è ottenuta con un moto rigido da $\mathbb{F}$ allora $W(\mathbb{F}^*) = W(\mathbb{F})$.

Sappiamo già [dove?] che $\mathbb{F}$ non cambia in seguito a traslazioni rigide, quindi possiamo scrivere

$\mathbb{F}^* = R\mathbb{F}$ con $R$ rotazione tale che $RR^T = \mathbb{I}$ e $\Vert R \Vert > 0$.

Ogni gradiente di deformazione $\mathbb{F}$ può essere scritto come il prodotto tra una rotazione e la sua componente puramente di deformazione $\mathbb{U}$ [non ha senso sta roba, o meglio non ne ha solo 1]. Quindi

$W(\mathbb{F}) = W(R\mathbb{U}) = W(\mathbb{U})$.

Possiamo ora definire il tensore di Cauchy-Green e il tensore di Green

$\mathbb{C} = \mathbb{F}^T\mathbb{F} = \mathbb{U}^T\mathbb{U}$

$\mathbb{G} = \frac{1}{2}(\mathbb{C}-\mathbb{I})$

che sono indipendenti da moti rigidi del corpo. [quindi W è più bella se definita in termini di questi due]

Per calcolare lo sforzo generato nel materiale dai tensori è conveniente introdurre i tensori di Piola-Kirchhoff, che ne sono la riscrittura in termini di coordinate lagrangiane

$(\mathbb{P}(\mathbf{s},t))_{\alpha,\beta} = \frac{\partial W}{\partial \mathbb{F}_{\alpha,\beta}} (\mathbf{s},t) =( \frac{\partial W}{\partial \mathbb{F}} (\mathbf{s},t))_{\alpha,\beta}$

$\mathbb{S}(\mathbf{s},t)=\frac{\partial W}{\partial\mathbb{G}}(\mathbf{s},t)=2\frac{\partial W}{\partial\mathbb{C}}(\mathbf{s},t)$

[]

$\int_{\partial\mathcal{P}_t} \mathbf{\sigma}\mathbf{n} \, dA = \int_{\partial\mathcal{P}} \mathbb{P}\mathbf{N} \, dA$

$\mathbb{P}(\mathbf{s},t) = \vert \mathbb{F}(\mathbf{s},t) \vert \mathbf{\sigma}(\mathbf{X}(\mathbf{s},t),t) \mathbb{F}^{-T}(\mathbf{s},t)$


Possiamo scrivere la legge di conservazione del momento in termini del primo tensore di P.K.

$M\frac{\partial^2\mathbf{X}}{\partial t^2} = \Nabla_\mathbf{s} \cdot \mathbb{P} + \mathbf{B}$

con 

$(\Nabla_\mathbf{s} \cdot \mathbb{P} )_{\alpha} = \frac{\partial\mathbb{P}_{\alpha,\beta}}{\partial s_\beta}$

\paragraph{Incomprimibilità}

Come già visto prima $\vert \mathbb{F} \vert = 1$.

[questo crea un casino e bisogna aggiungere un moltiplicatore di Lagrange associato alla condizione di incomprimibilità nell'equazione della pressione idrostatica wut?]

[modello neohookeano]

\paragraph{Materiali viscoelastici}

\subsection{Equazioni in forma forte}

Sia $\mathcal{B} \subseteq \Re^3$ una regione poligonale convessa sulla cui superficie $\partial\mathcal{B}$ è applicata una forza con densità per superficie $\mathbf{T}$.

Dati $\mathbf{X}_0$, $\mathbf{X}^{'}_0$ e $\mathbf{T}$ trovare $\mathbf{X}$ e $P$ tali che:
$$\frac{\partial^2\mathbf{X}}{\partial t^2} = \Nabla_\mathbf{s} \cdot \mathbb{P}$$
$$\mathbb{P} = \frac{\partial W}{\partial\mathbb{F}} - \vert\mathbb{F}\vert P\mathbb{F}^{-T} + \eta\vert\mathbb{F}\vert \frac{\partial}{\partial t}(\mathbb{F}\mathbb{F}^{-T} + (\mathbb{F}\mathbb{F}^{-T})^T)\mathbb{F}^{-T}$$
$$\vert\mathbb{F}\vert=1$$
$$\mathbf{X}(\mathbf{s},0)=\mathbf{X}_0(s)$$
$$\frac{\partial \mathbf{X}}{\partial t}(\mathbf{s},0)=\mathbf{X}^{'}_0(\mathbf{s})$$
$$\mathbb{P}\mathbf{N}=-\mathbf{T}$$

\subsection{Equazioni in forma debole}

Sia $S = \{ \mathbf{X} \in H^1(\mathcal{B})^d \cap W^{1,\infty}(\mathcal{B})^d \quad \texttt{t.c.} \quad \vert \boldsymbol {\nabla}_{\mathbf{s}}\mathbf{X}\vert = \vert \mathbb{F} \vert > 0, \mathbf{X}(\mathcal{B}) \subseteq \Omega \}$ lo spazio delle configurazioni ammesse e $P = L^2(\mathcal{B})^d$

Dati $\mathbf{T} \in H^{-\frac{1}{2}}(\partial\mathcal{B}^d$ e $\mathbf{X}_0,\mathbf{X}^{'}_0 \in S$ per ogni $t \in ]0,t[$ trovare $\mathbf{X}(t) \in S$ tale che 
$$\frac{d^2}{dt^2} < \mathbf{X}(t), \mathbf{Y} > + < \mathbb{P}(\mathbb{F}(t)), \Nabla_\mathbf{s} \mathbf{Y} = -\int_{\partial\mathcal{B}}\mathbf{T}(t)\cdot\mathbf{Y}\,dA$$
$$\mathbf{X}(\mathbf{s},0)=\mathbf{X}_0(\mathbf{s})$$
$$\frac{\partial\mathbf{X}}{\partial t}(\mathbf{s},0)=\mathbf{X}^{'}_0(\mathbf{s})$$

Le equazioni possono essere scritte in termini di operatori come fatto per le equazioni di N.S.

\paragraph{Stima dell'energia}

Presi $\mathbf{B}=0$ e $\mathbf{T}=0$ nelle equazioni allora
$$\frac{1}{2}\frac{d}{dt}\Vert\mathbf{U}(t)\Vert^2_{0,\mathcal{B}}+\frac{d}{dt}E[\mathbf{X}(t)]=0$$
[prendendo $\mathbf{Y}=\mathbf{U}$ nelle equazioni].

