% chapter1.tex

\chapter{Principi di Fluidodinamica ed Elasiticità}
%The Finite Element Immersed Boundary Method, L. Heltai ,Ph.D. Thesis , 2006, capitolo 1 19-35
La meccanica del continuo studia i corpi, siano essi solidi, liquidi o gassosi, come se il materiale che li compone si comportasse allo stesso modo ad ogni scala per quanto piccola, ignorando gli effetti dovuti alla struttura microscopica e alla natura atomica del materiale. In questo modo è possibile studiare questi corpi sfuttando gli strumenti del calcolo infinitesimale.

Le equazione che governano il comportamento dei corpi possono essere distinte in leggi di conservazione, che sono principi fisici generali, e da leggi costitutive che rappresentano il comportamento dello specifico materiale e variano da materiale a materiale e regolano la risposta alla deformazione.

\section{Introduzione}
%The Finite Element Immersed Boundary Method, L. Heltai ,Ph.D. Thesis , 2006, capitolo 1 19-35
%Mathematical Foundations of Elasticity, J.E. Mardsen, T.J.R. Hughes, 1994, Chapter "A POINT OF DEPARTURE"
Sia $\Omega \subset \mathbb{R}^{n}$ con $n=1,2,3$ una porzione dello spazio contenente un corpo continuo. Il punto $\mathbf{x}=(x_1,\dots,x_n) \in \Omega$ è detto punto spaziale e rappresenta un punto fisso nello spazio, e le sue componenti sono dette coordinate euleriane.
Sia $\mathcal{B} \subset \mathbb{R}^{d}$ con $d=1,2,3$ e $d\leq n$ la chiusura di un aperto con bordo a tratti liscio. $\mathcal{B}$ è detta configurazione di riferimento del corpo in esame. Il punto $\mathbf{s}=(s_1,\dots,s_d) \in \mathcal{B}$ è detto punto materiale e rappresenta un punto ben definito del corpo che si muove nello spazio, e le sue componenti sono dette coordinate lagrangiane. Una configurazione o deformazione di $\mathcal{B}$ è una mappa $\phi: \mathcal{B} \to \mathbb{R}^{n}$ sufficientemente liscia, che preserva l'orientamento e invertibile. 

La relazione tra i due sistemi di rifemento è data dalle funzioni
\begin{equation*}  
\phi:\mathcal{B} \times [0,T] \to \Omega
\end{equation*}
\begin{equation*}  
\mathbf{q}:\Omega \times [0,T] \to \mathcal{B}
\end{equation*}
per cui valgono le seguenti uguaglianze
\begin{equation*}  
\mathbf{x} = \phi(\mathbf{q}(\mathbf{x},t),t)
\end{equation*}
\begin{equation*}  
\mathbf{s} = \mathbf{q}(\phi(\mathbf{s},t),t)
\end{equation*}
Assumiamo che per ogni istante $t\in [0,T]$ le due funzioni $\mathbf{q}$ e $\phi$ siano invertibili e lipschitziane.

Osserviamo che $\phi(\mathbf{s},t)$ rappresenta la traiettoria nello spazio del punto materiale $\mathbf{s}$.

La matrice delle derivate parziali di $\phi$ 
\begin{equation*}
\mathbb{F} = D\phi
\end{equation*}
cioè
\begin{equation*}  
\mathbb{F}_{\alpha i} = (\Nabla_{s}\phi(\mathbf{s},t))_{\alpha i} = \frac{\partial \phi_{\alpha}(\mathbf{s},t)}{\partial s_{i}}
\end{equation*}
è detta gradiente di deformazione.

L'ipotesi che $\phi$ sia invertibile implica che $\mathbb{F}$ abbia determinante diverso da zero e quindi assumendo che, a meno di cambiamenti dell'ordine delle coordinate, $\vert\mathbb{F}\vert$ sia positivo al tempo $t=0$, questi rimarrà positivo per ogni tempo $t>0$.

La mappa $\phi$ preserva l'orientamento quindi $\mathbb{F}>0$. 

Il valore di $|\mathbb{F}|$ rappresenta il rapporto tra la misura della regione attorno al punto $\mathbf{s}$ nel riferimento lagrangiano e la misura della regione attorno al punto $\mathbf{x} = \phi(\mathbf{s},t)$ nel riferimento euleriano.

Il gradiente di deformazione $\mathbb{F}$ può essere decomposto in maniera univoca come 
\begin{equation*}
\mathbb{F}=\mathbb{R}\mathbb{U}=\mathbb{V}\mathbb{R}
\end{equation*}
dove $\mathbb{R}$ è una matrice ortogonale, e quindi rappresenta una rotazione, mentre $\mathbb{U}$ e $\mathbb{V}$ sono matrici simmetriche definite positive e sono dette rispettivamente tensore di stress destro e tensore di stress sinistro. 
Osserviamo noltre che 
\begin{equation*}
\mathbb{U}=\sqrt{\mathbb{F}^T\mathbb{F}}
\end{equation*}
e
\begin{equation*}
\mathbb{V}=\sqrt{\mathbb{F}\mathbb{F}^T}
\end{equation*}

Siano
\begin{equation*}
\mathbb{C}=\mathbb{F}^T\mathbb{F}=\mathbb{U}^2
\end{equation*}
il tensore destro di Cauchy-Green e
\begin{equation*}
\mathbf{b}=\mathbb{F}\mathbb{F}^T=\mathbb{V}^2
\end{equation*}
il tensore sinistro di Cauchy-Green.
Dato che $\mathbb{U}$ e $\mathbb{V}$ sono simili, simmetriche e definite positive allora hanno gli stessi autovalori $\lambda_1,\dots,\lambda_{d}$ reali e positivi, detti tensioni principali. I rispettivi autovettori sono detti direzioni principali. Il valore di $\lambda_i$ rappresenta lo stiramento nella direzione del rispettivo autovettore.

Osserviamo che la derivata temporale di grandezze espresse nei diversi sistemi di riferimento possiede, a seconda del sistema di riferimento, interpretazioni diverse.
Nel caso del riferimento lagrangiano rappresenta la variazione di una quantità fisica nel punto in esame, nel caso del riferimento euleriano non ha un'interpretazione fisica diretta perchè oltre alla variazione della grandezza bisogna anche considerare le variazioni indotte dal moto del corpo.

Sia $\Psi(\mathbf{x},t)$ una grandezza espresse in coordinate euleriane, allora

\begin{multline*}
\frac{d\Psi(\mathbf{x},t)}{dt} = \\
\frac{\partial \Psi(\phi(\mathbf{s},t))}{\partial t} = 
\frac{\partial \Psi(\mathbf{x},t)}{\partial t} + \frac{\partial\phi(\mathbf{s},t)}{\partial t}\vert_{\mathbf{s} = \mathbf{q}(\mathbf{x},t)}
\cdot \Nabla\Psi(\mathbf{x},t) = \\
\frac{\partial \Psi(\mathbf{x},t)}{\partial t} + \mathbf{u}(\mathbf{x},t)\cdot \Nabla\Psi(\mathbf{x},t)
\end{multline*}

dove 
\begin{equation*}
\mathbf{u}(\mathbf{x},t) = \frac{\partial\phi(\mathbf{s},t)}{\partial t}|_{\mathbf{s} = \mathbf{q}(\mathbf{x},t)}
\end{equation*}
è la velocità della particella del corpo che al tempo $t$ si trova nella posizione $\mathbf{x}$.

L'operatore
\begin{equation*}
\frac{D\Psi}{Dt} = \frac{\partial\Psi}{\partial t} + \mathbf{u}\cdot\Nabla\Psi
\end{equation*}
è detto derivata materiale e rendo esplicito come nella variazione nel tempo di una grandezza espressa in coordinate euleriane bisogna considerare anche le variazioni dovute al moto del corpo.

Un moto di $\mathcal{B}$ è una famiglia di configurazioni dipendente dal tempo
\begin{equation*}
\phi: \mathcal{B} \times [0,T] \to R^{n}
\end{equation*}
La velocità del punto materiale $\mathbf{s}$ è data da
\begin{equation*}
\mathbf{V}(\mathbf{s},t) = \frac{\partial \phi}{\partial t}(\mathbf{s},t)
\end{equation*}
mentre la velocità del punto spaziale $\mathbf{x}=\phi(\mathbf{s},t)$ è data da
\begin{equation*}
\mathbf{v}(\mathbf{x},t) = \mathbf{V}(\mathbf{s},t)
\end{equation*}
L'accelerazione è data da
\begin{multline*}
\mathbf{A}(\mathbf{s},t) = \frac{\partial^2 \phi}{\partial t^2}(\mathbf{s},t) = \frac{\partial \mathbf{V}}{\partial t}(\mathbf{s},t) = \\ \frac{\partial \mathbf{v}}{\partial t}(\phi(\mathbf{s},t),t)+\mathbf{v}(\phi(\mathbf{s},t),t)\cdot\Nabla\mathbf{v}(\phi(\mathbf{s},t),t)
\end{multline*}
Più in generale se $Q$ è una grandezza materiale e $q$ la relativa grandezza spaziale allora
\begin{equation*}
\frac{\partial Q}{\partial t} = \frac{\partial q}{\partial t} + \mathbf{u}\cdot\Nabla q = \frac{Dq}{Dt}
\end{equation*}

\subsection{Il teorema del trasporto}

Il teorema del trasporto di Reynolds permette di studiare l'evoluzione delle quantità fisiche di una regione di corpo in movimento.

\begin{theorem}[del Trasporto di Reynolds]
Dato un campo scalare o vettoriale $\Psi(\mathbf{x},t)$ e una regione regolare di materiale $\mathcal{P}\subset\mathcal{B}$, sia $\mathcal{P}_{t} = \phi(\mathcal{P},t)$ la stessa regione in coordinate euleriane al tempo $t$ allora per ogni $t\in (0,T)$ vale:
\begin{multline}
\frac{d}{dt}\int_{\mathcal{P}_{t}}\Psi(\mathbf{x},t)\,d\mathbf{x} =
 \int_{\mathcal{P}_{t}}\left(\frac{D\Psi(\mathbf{x},t)}{Dt}+\Psi(\mathbf{x},t)\Nabla\cdot\mathbf{u}(\mathbf{x},t)\right)\,d\mathbf{x} = \\
\int_{\mathcal{P}_{t}}\frac{\partial\Psi(\mathbf{x},t)}{\partial t}\,d\mathbf{x}+ \int_{\partial\mathcal{P}_{t}}\Psi(\mathbf{x},t)\mathbf{u}(\mathbf{x},t)\cdot\mathbf{n}\,d\mathbf{a}
\end{multline}
\end{theorem}
\begin{lemma}
Sia $J(\mathbf{x},t)=|\mathbb{F}|(\mathbf{x},t)$ il determinante del gradiente di deformazione allora
\begin{equation}
\frac{\partial}{\partial t}J = (\Nabla \cdot \mathbf{u})J
\end{equation}
\end{lemma}
\begin{proof}
Osserviamo che 
\begin{equation}
\frac{\partial}{\partial t}\phi(\mathbf{x},t) = \mathbf{u}(\phi(\mathbf{x},t))
\end{equation}
Scriviamo esplicitamente le componenti di $\phi$ come
\begin{equation*}
\phi(\mathbf{s},t)=(\chi(\mathbf{s},t),\eta(\mathbf{s},t),\xi(\mathbf{s},t))
\end{equation*}
e le componenti di $\mathbf{u}$ come
\begin{equation*}
\mathbf{u}(\mathbf{x},t)=(u(\mathbf{x},t),v(\mathbf{s},t),w(\mathbf{s},t))
\end{equation*}
Sfruttando la proprietà per cui il determinante è multilineare per righe e per colonne si può differenziare ottenedo che
\begin{multline*}
\frac{\partial}{\partial t}J = \\
\begin{vmatrix}
\frac{\partial}{\partial t}\frac{\partial \chi}{\partial s_1} & \frac{\partial \eta}{\partial s_1} & \frac{\partial \xi}{\partial s_1} \\
\frac{\partial}{\partial t}\frac{\partial \chi}{\partial s_2} & \frac{\partial \eta}{\partial s_2} & \frac{\partial \xi}{\partial s_2} \\
\frac{\partial}{\partial t}\frac{\partial \chi}{\partial s_3} & \frac{\partial \eta}{\partial s_3} & \frac{\partial \xi}{\partial s_3} \\
\end{vmatrix}
+
\begin{vmatrix}
\frac{\partial \chi}{\partial s_1} & \frac{\partial}{\partial t}\frac{\partial \eta}{\partial s_1} & \frac{\partial \xi}{\partial s_1} \\
\frac{\partial \chi}{\partial s_2} & \frac{\partial}{\partial t}\frac{\partial \eta}{\partial s_2} & \frac{\partial \xi}{\partial s_2} \\
\frac{\partial \chi}{\partial s_3} & \frac{\partial}{\partial t}\frac{\partial \eta}{\partial s_3} & \frac{\partial \xi}{\partial s_3} \\
\end{vmatrix}
+
\begin{vmatrix}
\frac{\partial \chi}{\partial s_1} & \frac{\partial \eta}{\partial s_1} & \frac{\partial}{\partial t}\frac{\partial \xi}{\partial s_1} \\
\frac{\partial \chi}{\partial s_2} & \frac{\partial \eta}{\partial s_2} & \frac{\partial}{\partial t}\frac{\partial \xi}{\partial s_2} \\
\frac{\partial \chi}{\partial s_3} & \frac{\partial \eta}{\partial s_3} & \frac{\partial}{\partial t}\frac{\partial \xi}{\partial s_3} \\
\end{vmatrix}
\end{multline*}
Osservato che
\begin{multline*}
\frac{\partial}{\partial t}\frac{\partial \chi}{\partial s_1}(\mathbf{s},t) =
\frac{\partial}{\partial s_1}\frac{\partial \chi}{\partial t}(\mathbf{s},t) = 
\frac{\partial}{\partial s_1}u(\phi(\mathbf{s},t)),t) = \\ 
\frac{\partial u}{\partial \chi}(\mathbf{x},t)\frac{\partial\chi}{\partial s_1}(\mathbf{s},t)
+ \frac{\partial u}{\partial \eta}(\mathbf{x},t)\frac{\partial\eta}{\partial s_2}(\mathbf{s},t)
+ \frac{\partial u}{\partial \xi}(\mathbf{x},t)\frac{\partial\xi}{\partial s_3}(\mathbf{s},t)
\end{multline*}
e ricordando che
\begin{equation*}%[Formula di Voss-Weyl]
\Nabla\cdot\mathbf{u} = \sum_{i=1}^{3}{\frac{1}{J}\frac{\partial(Ju_i) }{\partial s_i}}
\end{equation*}
sostituendo nell'equazione precedente si ottiene
\begin{equation*}
\frac{\partial u}{\partial x} J + \frac{\partial u}{\partial y} J +\frac{\partial u}{\partial z} J = (\Nabla \cdot \mathbf{u}) J
\end{equation*}
\end{proof}
\begin{proof}[Dimostrazione (Teorema del trasporto)]
Effettuando un cambiamento di variabile per passare dal riferimento euleriano a lagrangiano, differenziando poi sotto il segno di integrale ora che il dominio di integrazione non dipende più da $t$ e sfruttando il lemma, osservato che $J$ rappresenta proprio il fattore introdotto dal cambio di variabile, si ottiene la tesi.
\begin{multline*}
\frac{d}{dt}\int_{\mathcal{P}_{t}}\Psi(\mathbf{x},t)\,d\mathbf{x} =
\frac{d}{dt}\int_{\mathcal{P}}\Psi(\phi(\mathbf{s},t),t)J(\mathbf{s},t)\,d\mathbf{s} = \\
\int_{\mathcal{P}}\left(\frac{d\Psi(\phi(\mathbf{s},t),t)}{d t}J(\mathbf{s},t)+\frac{\partial  J(\mathbf{s},t)}{\partial t}\Psi(\phi(\mathbf{s},t),t)\right)\,d\mathbf{s} = \\
\int_{\mathcal{P}}\left(\frac{\partial\Psi(\phi(\mathbf{s},t),t)}{\partial t}J(\mathbf{s},t)+\Nabla\cdot\mathbf{u}(\phi(\mathbf{s},t),t)\Psi(\phi(\mathbf{s},t),t) J(\mathbf{s},t)\right)\,d\mathbf{s} = \\
\int_{\mathcal{P}_t}\left(\frac{D\Psi(\mathbf{x},t)}{D t}+\Nabla\cdot\mathbf{u}(\mathbf{x},t))\Psi(\mathbf{x},t)\right)\,d\mathbf{x} 
\end{multline*}
\end{proof}

Come vedremo nella sezione successiva è possibile ricavare le equazioni che descrivono le leggi di conservazione applicando il teorema del trasporto, ad esempio la conservazione del volume si può scrivere come
\begin{equation}
0 = \frac{d}{dt}\int_{\mathcal{W}_t}\,dV = \int_{\mathcal{W}_t} \Nabla \cdot \mathbf{u}\,dV
\end{equation}
e, dato che deve valere per ogni $t$ e per ogni $\mathcal{W}_t$, questo implica che $\Nabla \cdot \mathbf{u} = 0$

\section{Cenni di Fluidodinamica}

%A Mathematical Introduction to Fluid Mechanics, A. Chorin, J. E. Marsden, 1992 Springer-Verlag Publishing Company Inc., terza edizione (forse preprint quarta) Capitolo 1 1-46
La fluidodinamica studia e descrive il movimento dei fluidi. Le equazioni che governano questo movimento sono ricavate a partire dalle leggi di conservazione della massa, del momento e dell'energia. Nel caso di fluidi perfetti si ottengono le equazioni di Eulero, mentre nel caso di fluidi viscosi si ottengono le equazioni di Navier-Stokes.

\subsection{Le equazioni di Eulero}

Sia $\mathcal{D} \subseteq \mathbb{R}^{2,3}$ una porzione di spazio contenente un fluido, $\mathbf{x} \in \mathcal{D}$ un punto del fluido in coordinate euleriane, $\mathbf{u}(\mathbf{x},t)$ la velocità del fluido nel punto $\mathbf{x}$ al tempo $t$. Per ogni tempo $t$, $\mathbf{u}$ è un campo vettoriale su $\mathcal{D}$ ed è detto campo delle velocità del fluido. Sia $\rho(\mathbf{x},t)$ la densità del fluido nel punto $\mathbf{x}$ al tempo $t$.  L'esistenza di $\rho$ è detta ipotesi del continuo ed equivale a considerare il corpo come un oggetto senza soluzione di continuità trascurando la natura molecolare del fluido. Questa ipotesi rimane accettabile fintanto che si studiano i fenomeni macroscopici che avvengono nel fluido. Sia $\mathcal{W} \subseteq \mathcal{D}$ una parte dello spazio occupato dal fluido e
\begin{equation*}
m(\mathcal{W},t)= \int_{\mathcal{W}}\rho(\mathbf{x},t) \, dV
\end{equation*}
la massa del fluido contenuto in $\mathcal{W}$ al tempo $t$. Si supponga sempre inoltre che le funzioni usate abbiano sufficiente regolarità per dare senso alle espressioni scritte. 

Le equazioni di Eulero sono ricavate a partire del principio di conservazione della massa (\emph{la massa non si crea e non si distrugge}), del momento (\emph{seconda legge di Newton}) e dell'energia (\emph{l'energia non si crea e non si distrugge}).

\paragraph{La conservazione della massa}
Nell'ipotesi che la massa del fluido si conservi è necessario che le uniche variazioni di massa per ogni porzione di fluido siano dovute al solo fluido che entra o esce dalla data porzione. Sia $\partial \mathcal{W}$ il bordo, supposto liscio, di  $\mathcal{W}$ e $\mathbf{n}(\mathbf{x})$ la normale uscente a $\partial \mathcal{W}$ nel punto $\mathbf{x}$ allora la massa del fluido uscente da $\mathcal{W}$ per unità di area $dA$ per unità di tempo è data da $\rho \mathbf{u}\cdot \mathbf{n}$.
\begin{equation*}  
\begin{split}
\frac{d}{dt} m(\mathcal{W},t) &= \frac{d}{dt}  \int_{\mathcal{W}}\rho(\mathbf{x},t) \, dV \\
                              &= \int_{\mathcal{W}}\frac{\partial}{\partial t}\rho(\mathbf{x},t) \, dV \\
                              &= - \int_{\partial \mathcal{W}}\rho(\mathbf{x},t) \mathbf{u}(\mathbf{x},t) \cdot \mathbf{n}(\mathbf{x}) \, dV
\end{split}
\end{equation*}
che, applicando il teorema della divergenza al secondo membro, diventa
\begin{equation*}
\int_{\mathcal{W}}\frac{\partial}{\partial t}\rho(\mathbf{x},t) \, dV = -\int_{\mathcal{W}}\Nabla \cdot \left (\rho(\mathbf{x},t) \mathbf{u}(\mathbf{x},t) \right ) \, dV
\end{equation*}
e quindi
\begin{equation}\label{eq:conservazione massa integrale}
\boxed{
\int_{\mathcal{W}}\left [ \frac{\partial}{\partial t}\rho(\mathbf{x},t) + \Nabla \cdot (\rho(\mathbf{x},t) \mathbf{u}(\mathbf{x},t)) \right ]\, dV = 0
}
\end{equation}
L'equazione \ref{eq:conservazione massa integrale} è detta legge della conservazione della massa in forma integrale.

Dovendo essere vera per ogni $\mathcal{W}$ l'equazione \ref{eq:conservazione massa integrale} è equivalente a 
\begin{equation}\label{eq:conservazione massa differenziale}
\boxed{
\frac{\partial \rho}{\partial t} + \Nabla \cdot (\rho \mathbf{u}) = 0
}
\end{equation}
Questa è la forma differenziale della legge di conservazione della massa, altrimenti detta equazione di continuità.

Lo stesso risultato si può ottenere applicando il teorema del trasporto
\begin{multline*}
0 = \frac{d}{dt}\int_{\mathcal{W}_t}\rho\,dV = \int_{\mathcal{W}_t}\left(\frac{D\rho}{D t}+\rho(\Nabla\cdot\mathbf{u})\right)\,dV = \\
\int_{\mathcal{W}_t}\left(\frac{\partial\rho}{\partial t}+\mathbf{u}\cdot\Nabla\rho+\rho(\Nabla\cdot\mathbf{u})\right)\,dV = 
\int_{\mathcal{W}_t}\left(\frac{\partial\rho}{\partial t}+\Nabla(\rho\mathbf{u})\right)\,dV = 
\end{multline*}

\paragraph{La conservazione del momento}
Sia
\begin{equation*}
\mathbf{x}(t) = (x(t),y(t),z(t))
\end{equation*}
la traiettoria seguita da una particella del fluido e
\begin{equation*}
\mathbf{u}(t) = (\dot{x}(t),\dot{y}(t),\dot{z}(t)) = \mathbf{u}(\mathbf{x}(t)) = \frac{d\mathbf{x}}{dt}(t)
\end{equation*}
la sua velocità allora l'accelerazione della particella sarà data da
\begin{equation*} 
\mathbf{a}(\mathbf{x}(t)) = \frac{d^2\mathbf{x}}{dt^2} (t) = \frac{\partial  \mathbf{u}}{\partial x}\dot{x} + \frac{\partial  \mathbf{u}}{\partial y}\dot{y} + \frac{\partial  \mathbf{u}}{\partial z}\dot{z} + \frac{\partial \mathbf{u}}{\partial t}
= \nabla  \mathbf{u} \cdot  \mathbf{u} + \frac{\partial \mathbf{u}}{\partial t} = \frac{D \mathbf{u}}{Dt}
\end{equation*}

Le forze che agiscono su di un corpo continuo possono essere di due soli tipi: le forze di stress e le forze esterne o esogene. Le forze di stress sono le forze applicate su una superficie interna al corpo dal resto del corpo. Le forze esterne sono invece forze applicate per unità di volume da agenti esterni al corpo, come il campo gravitazionale o il campo magnetico. Un fluido si dice ideale se esiste una funzione $p(\mathbf{x},t)$ detta pressione per cui le forze di stress sono del tipo
\begin{equation*}
\mathbf{F}\,dA = p(\mathbf{x},t)\mathbf{n}
\end{equation*}
cioè sono forze normali alla superficie. In particolare l'assenza di componenti tangenziali delle forze di stress impedisce l'instaurarsi o il cessare di movimenti rotatori nel fluido.

Sia $\mathcal{W}$ una regione del fluido, allora la forza di stress applicata su $\mathcal{W}$ dal resto del corpo è data da
\begin{equation*}
\mathbf{S}_{\partial\mathcal{W}} = - \int_{\partial\mathcal{W}} p\mathbf{n}\,dA
\end{equation*}
Scelta una direzione $\mathbf{e}$ applicando il teorema della divergenza si ottiene
\begin{equation*}
\mathbf{e}\cdot\mathbf{S}_{\partial\mathcal{W}} = - \int_{\partial\mathcal{W}} p\mathbf{e}\cdot\mathbf{n}\,dA
                                                  = - \int_{\mathcal{W}} \nabla p \cdot \mathbf{e} \, dV
\end{equation*}
e quindi, data l'arbitrarietà di $\mathbf{e}$
\begin{equation*}
\mathbf{S}_{\partial\mathcal{W}} = - \int_{\mathcal{W}} \nabla p \, dV
\end{equation*}

Sostituendo nella seconda legge di Newton a cui si è aggiunto il termine relativo alle forze esterne si ottiene
\begin{align*}
\mathbf{F} &= m\mathbf{a} \\
\Rightarrow \int_{\mathcal{W}}\left(-\nabla p + \rho\mathbf{b}\right)\, dV &= \int_{\mathcal{W}}\rho\frac{D\mathbf{u}}{Dt}\, dV
\end{align*}

\begin{equation}\label{eq:conservazione momento differenziale}
\boxed{
\rho \frac{\partial \mathbf{u}}{\partial t} = - \rho \mathbf{u} \cdot \nabla \mathbf{u} - \nabla p + \rho \mathbf{b}
}
\end{equation}
che è la legge di conservazione del momento in forma differenziale.

La forma integrale della legge di conservazione del momento può essere ricavata direttamente dalla forma differenziale.
\begin{equation}
\boxed{
\frac{d}{dt}\int_{\mathcal{W}}\rho \mathbf{u} = - \int_{\partial\mathcal{W}}(p\mathbf{n}+\rho\mathbf{u}(\mathbf{u} \cdot \mathbf{n})) \, dA + \int_{\mathcal{W}}\rho\mathbf{b}\, dV
}  
\end{equation}
La quantità
\begin{equation*}
p\mathbf{n}+\rho\mathbf{u}(\mathbf{u}\cdot\mathbf{n})
\end{equation*}
è detta flusso di momento per unità di area attraverso $\partial\mathcal{W}$.

E' possibile ricavare la forma integrale della legge di conservazione del momento senza passare dalla forma differenziale, in modo che sia necessario richiedere una minore regolarità alle funzioni coinvolte, sfruttando il teorema del trasporto.
Sia $\phi(\mathbf{s},t)$ la traiettoria seguita dal punto materiale che si trova in posizione $\mathbf{x}$ al tempo $t=0$. Si ipotizzi che $\phi$ sia regolare e invertibile per $t$ fissato. Sia $\phi_t:\mathbf{s}\mapsto\phi (\mathbf{s},t)$ la mappa che associa ad ogni punto la posizione che avrà al tempo $t$. $\phi$ è detta mappa del flusso del fluido. Indichiamo con $\mathcal{W}_t = \phi_t(\mathcal{W})$ il volume $\mathcal{W}$ trasportato dal fluido.

La legge di conservazione del momento si scrive quindi come
\begin{equation}\label{eq:conservazione momento integrale}
\frac{d}{dt} \int_{\mathcal{W}_t} \rho \mathbf{u} \, dV = S_{\partial \mathcal{W}_t} +  \int_{\mathcal{W}_t} \rho \mathbf{b} \, dV
\end{equation}

Questa è equivalente alla forma differenziale se le funzioni sono sufficientemente regolari.

Applicando il teorema del trasporto
\begin{multline*}
\frac{d}{dt} \int_{\mathcal{W}_t} \rho \mathbf{u} \, dV = \int_{\mathcal{W}}\left(\frac{D(\rho \mathbf{u})}{D t}+\Nabla\cdot\mathbf{u}(\rho \mathbf{u}) \right)\, dV = \\
\int_{\mathcal{W}}\left(\frac{\partial(\rho \mathbf{u})}{\partial t}+\Nabla\cdot(\rho \mathbf{u})+\Nabla\cdot\mathbf{u}(\rho \mathbf{u}) \right)\, dV = \\
\end{multline*}
Sfruttando ora la legge di conservazione della massa
\begin{equation*}
\frac{D\rho}{D t}+\rho\Nabla\cdot\mathbf{u} = \frac{\partial\rho}{\partial t} + \Nabla\cdot(\rho\mathbf{u})
\end{equation*}
si ottiene
\begin{equation*}
\frac{d}{dt} \int_{\mathcal{W}_t} \rho \mathbf{u} \, dV = \int_{\mathcal{W}_t} \rho \frac{D\mathbf{u}}{Dt}\, dV
\end{equation*}

Questo ragionamento è in realtà valido per ogni funzione
\begin{theorem}
$\forall f:\mathbb{R}^3\times[0,T]\mapsto \mathbb{R}$ vale la seguente eguaglianza
\begin{equation*}
\frac{d}{dt}\int_{\mathcal{W}_t}\rho f\, dV = \int_{\mathcal{W}_t}\rho\frac{Df}{Dt}\,dV
\end{equation*}
\end{theorem}
Osserviamo che si tratta di una forma particolare del teorema del trasporto.

Essendo le funzioni continue e $\mathcal{W}$ qualunque allora \ref{eq:conservazione momento integrale} è equivalente a \ref{eq:conservazione momento differenziale}.

Sfruttando il lemma \ref{lemma:incomprimibile} possiamo caratterizzare i fluidi incomprimibili infatti
\begin{equation*}
\volume(\mathcal{W}_t) = \int_{\mathcal{W}_t} \, dV
\end{equation*}
quindi per un fluido incomprimibile
\begin{equation*}
0 = \frac{d}{dt} \int_{\mathcal{W}_t} \, dV  = \int_{\mathcal{W}_t} \Nabla \cdot \mathbf{u} \, dV
\end{equation*}
Quindi un fluido è incomprimibile se e solo se 
\begin{equation*}
\Nabla \cdot \mathbf{u}=0
\end{equation*}
o anche se e solo se
\begin{equation*}
J \equiv 1
\end{equation*}
Osserviamo che
\begin{equation*}
0 = \frac{\partial \rho}{\partial t} + \Nabla\cdot(\rho \mathbf{u}) = \frac{D\rho}{Dt} + \rho\Nabla\cdot \mathbf{u}
\end{equation*}
quindi un fluido è incomprimibile se e solo se
\begin{equation*}
\frac{D \rho}{Dt} = 0
\end{equation*}
cioè se e solo se la densità del fluido è costante seguendo il flusso.
Un fluido si dice omogeneo se $\rho(\mathbf{x},t)=C_t \quad \forall \mathbf{x}$, è anche incomprimibile se $C_t=C \quad \forall t$

Diamo ora un'altra caratterizzazione della legge di conservazione della massa sfruttando il teorema del trasporto.
$$\frac{d}{dt}\int_{\mathcal{W}_t}\rho\,dV=0$$
e quindi
$$\int_{\mathcal{W}_t}\rho(\mathbf{x},t)\,dV=\int_\mathcal{W}\rho(\mathbf{x},0)\,dV$$
Effettuando un cambio di variabile
$$\int_{\mathcal{W}}\rho(\phi(\mathbf{x},t),t)J(\mathbf{x},t)\,dV=\int_\mathcal{W}\rho(\mathbf{x},0)\,dV$$
e quindi
\begin{equation}
  \boxed{
  \rho(\phi(\mathbf{x},t),t)J(\mathbf{x},t)=\rho(\mathbf{x},0)\,dV
  }
\end{equation}
In particolare si nota che un fluido omogeneo ma non incomprimibile non rimane omogeneo col passare del tempo.

\paragraph{La conservazione dell'energia}
Sia
\begin{equation*}
E_{\text{kinetic}} = \frac{1}{2} \int_{\mathcal{W}} \rho \Vert \mathbf{u} \Vert^2 \, dV
\end{equation*}
l'energia cinetica del fluido contenuto in $\mathcal{W}$, $E_{\text{internal}}$ l'energia interna dovuta a fenomeni microscopici tra le molecole del fluido e $$E_{\text{total}}=E_{\text{kinetic}}+E_{\text{internal}}$$ l'energia totale.
La variazione dell'energia cinetica nella porzione di fluido in movimento $\mathcal{W}_t$ calcolata grazie al teorema del trasporto è pari a
\begin{align*}
\frac{d}{dt} E_{\text{kinetic}} &= \frac{d}{dt} \frac{1}{2} \int_{\mathcal{W}} \rho \Vert \mathbf{u} \Vert^2 \, dV \\
                                &= \frac{1}{2}  \int_{\mathcal{W}_t} \rho \frac{D\Vert \mathbf{u} \Vert^2}{Dt} \, dV \\
                                &=  \int_{\mathcal{W}_t} \rho ( \mathbf{u} \cdot ( \frac{\partial \mathbf{u}}{\partial t} + (\mathbf{u} \cdot \nabla) \mathbf{u} )) \, dV
\end{align*}
risultato ottenuto osservando che
\begin{multline*}
\frac{1}{2}\frac{D}{Dt}\Vert\mathbf{u}\Vert^2 =
\frac{1}{2}\frac{\partial}{\partial t}(u^2+v^2+w^2)+ \\
\frac{1}{2}\left(u\frac{\partial}{\partial x}(u^2+v^2+w^2)+
v\frac{\partial}{\partial y}(u^2+v^2+w^2)+
w\frac{\partial}{\partial z}(u^2+v^2+w^2)\right) = \\
u\frac{\partial u}{\partial t} + v\frac{\partial v}{\partial t} +w\frac{\partial w}{\partial t} +
u\left(u\frac{\partial u}{\partial x}
+v\frac{\partial v}{\partial x}+w\frac{\partial w}{\partial x}\right)
+ \\ v\left(u\frac{\partial u}{\partial y}
+v\frac{\partial v}{\partial y}+w\frac{\partial w}{\partial y}\right)
 +
w\left(u\frac{\partial u}{\partial z}
+v\frac{\partial v}{\partial z}+w\frac{\partial w}{\partial z}\right) = \\
\mathbf{u}\frac{\partial\mathbf{u}}{\partial t}+\mathbf{u}\cdot(\mathbf{u}\cdot\Nabla\mathbf{u})
\end{multline*}


Consideriamo ora il caso in cui il fluido sia incomprimibile e che tutta l'energia del fluido sia nella forma di energia cinetica (è sufficiente che l'energia interna sia costante). Allora la variazione dell'energia cinetica è dovuta al lavoro compiuto dalla pressione e dalle forze esterne

\begin{equation*}
\frac{d}{dt} E_{\text{kinetic}} = - \int_{\partial \mathcal{W}_t} p \mathbf{u} \cdot \mathbf{n} \, dA + \int_{\mathcal{W}_t} \rho \mathbf{u} \cdot \mathbf{b} \, dV
\end{equation*}

e quindi usando il teorema della divergenza e tenendo conto che
\begin{equation*}
\nabla \cdot \mathbf{u} = 0
\end{equation*}
si ottiene
\begin{align*}
\int_{\mathcal{W}_t} \rho \left ( \mathbf{u} \cdot \left ( \frac{\partial \mathbf{u}}{\partial t} + \mathbf{u} \cdot \Nabla \mathbf{u} \right )\right ) \, dV
 &= - \int_{\mathcal{W}_t} ( \Nabla \cdot (p\mathbf{u}) - \rho \mathbf{u} \cdot \mathbf{b}) \, dV \\
 &=  - \int_{\mathcal{W}_t} ( \nabla p  \cdot \mathbf{u} - \rho \mathbf{u} \cdot \mathbf{b}) \, dV
\end{align*}
Osserviamo che se si suppone che tutta l'energia del corpo sia sotto forma di energia cinetica allora il fluido è incomprimibile (oppure $p\equiv 0$).
 
Si ottengono cosi le equazioni di Eulero per i fluidi incomprimibili
\begin{equation}
\boxed{
\begin{aligned}
\rho \frac{D\mathbf{u}}{Dt} &= - \nabla p + \rho \mathbf{b} \\
\frac{D \rho }{Dt} &= 0 \\
\nabla \cdot \mathbf{u} &= 0
\end{aligned}
}
\end{equation}
a cui va aggiunta la condizione al bordo 
\begin{equation}\mathbf{u} \cdot \mathbf{n} = 0 \quad \text{su} \quad \partial \mathcal{D}
\end{equation}

%MARDSEN
\paragraph{Leggi di Conservazione}
\subparagraph{Conservazione della Massa}
Sia $\boldsymbol{\phi}(\mathbf{x},t)$ un moto di $\mathcal{B}$ e $\rho(\mathbf{x},t)$ la densità di massa del corpo deformato e $\rho_{\mathcal{B}}(\mathbf{X})$ la densità di massa del corpo non deformato. 
Sia $\mathcal{U}\in\mathcal{B}$ una porzione del corpo $\mathcal{B}$ e  $\mathcal{U}_t=\boldsymbol{\phi}(\mathcal{U},t)$ la stessa porzione al tempo $t$. Allora 
$$\int_{\mathcal{U}}\rho_{\mathcal{B}}(\mathbf{X})\, d\mathbf{X} = \int_{\mathcal{U}_t}\rho(\mathbf{x},t)\, d\mathbf{x} = $$
[...] e quindi
$$\int_{\mathcal{U}_t}\frac{\partial\rho}{\partial t}(\mathbf{x},t)\, d\mathbf{x} =  -\int_{\partial\mathcal{U}_t} \mathbf{J} \cdot \mathbf{n} \, da$$
\subparagraph{Conservazione del Momento}
Sia $\mathbf{b}(\mathbf{x},t)$ la forza applicata al corpo $\mathcal{B}$ per unità di massa, e sia $\boldsymbol{\tau}(\mathbf{x},t)$ la forza di superficie applicata per unità di superficie. La coppia $(\mathbf{b},\boldsymbol{\tau})$ è detto carico applicato al corpo. Sia $\mathbf{t}(\mathbf{x},t,\mathbf{n})$ la forza di stress interna al corpo per unità di area nella direzione $\mathbf{n}$. La seconda legge di Newton diventa quindi
$$\frac{d}{dt}\int_{\mathcal{U}_t}\rho\mathbf{v}\, d\mathbf{x} = \int_{\partial \mathcal{U}_t}\mathbf{t}\, da + \int_{\mathcal{U}_t}\rho\mathbf{b}\, d\mathbf{x} $$.
Per il teoreme di Cauchy se vale la legge di conservazione del momento allora $\mathbf{t}$ dipende linearmente da  $\mathbf{n}$, e quindi esiste un tensore $\boldsymbol{\sigma}$ tale che 
$$\mathbf{t}(\mathbf{x},t,\mathbf{n})= \boldsymbol{\sigma}(\mathbf{x},t)\mathbf{n}$$. Questo tensore è detto tensore degli stress di Cauchy. Sfruttando il teorema della divergenza di ottiene
$$\frac{d}{dt}\int_{\mathcal{U}_t}\rho\mathbf{v}\, d\mathbf{x} = \int_{\mathcal{U}_t}\Nabla\cdot\boldsymbol{\sigma}+\rho\mathbf{b}\, d\mathbf{x} $$.
Data l'arbitrarietà di ${\mathcal{U}_t}$ si ottiene l'equazione del moto di Cauchy
$$\rho\frac{d\mathbf{v}}{dt}=\Nabla\cdot\boldsymbol{\sigma}+\rho\mathbf{b}$$. 

%HELTAI
Le equazioni di conservazione possono essere ricavate applicando il teorema del trasporto.

\paragraph{Conservazione del volume}
Applicando il teorema di Reynolds alla funzione $\Phi \equiv 1$ si ottiene la legge di conservazione del volume
$$\frac{d}{dt}\int_{\mathcal{P}_t}\,d\mathbf{x}=\int_{\mathcal{P}_t}\Nabla\cdot\mathbf{u}\,d\mathbf{x}=\int_{\partial\mathcal{P}_t}\mathbf{u}\cdot\mathbf{n}\,d\mathbf{x} $$.
Se il materiale è incomprimibile allora $\Nabla\cdot\mathbf{u}=0$.

\paragraph{Conservazione della massa}
Sia $M(\mathbf{s})$ la densità di massa del punto materiale $\mathbf{s}$ e 
$$\rho(\mathbf{x},t)=M(\mathbf{X}(\mathbf{x},t))$$
la densità di massa in coordinate euleriane.
Applicando il principio di conservazione della massa si ottiene
$$\frac{d}{dt}\int_{\mathcal{P}_t}\rho\,d\mathbf{x}=\int_{\mathcal{P}_t}\frac{D\rho}{Dt}+\rho\Nabla\cdot\mathbf{u}\,d\mathbf{x}=0$$.
Se inoltre il materiale è incomprimibile allora 
$$\frac{D\rho}{Dt}(\mathbf{x},t)=0 $$.
e applicando il teorema di Reynold alla funzione $\rho\Phi$ si ottiene
$$\frac{d}{dt}\int_{\mathcal{P}_t}\rho\Phi\,d\mathbf{x}=\int_{\mathcal{P}_t}\frac{D(\rho\Phi)}{Dt}+\rho\Phi\Nabla\cdot\mathbf{u}\,d\mathbf{x}=\int_{\mathcal{P}_t}(\frac{D\rho}{Dt}+\rho\Nabla\cdot\mathbf{u})\Phi+\rho\frac{D\Phi}{Dt}\,d\mathbf{x}=\int_{\mathcal{P}_t}\rho\frac{D\Phi}{Dt}\,d\mathbf{x}$$.
Quindi in un materiale incomprimibile la densità di massa si comporta come una costante per quanto riguarda l'operazione di derivazione materiale.

\paragraph{Conservazione della quantità di moto}
Sia
$$\mathbf{p}(\mathcal{P},t)=\int_{\mathcal{P}_t}\rho\mathbf{u}\,d\mathbf{x}$$
la quantità di moto della porzione di corpo $\mathcal{P}$
e
$$\mathbf{L}(\mathcal{P},t)=\int_{\mathcal{P}_t}\rho\mathbf{x}\times\mathbf{u}\,d\mathbf{x}$$
il momento angolare.
Applicando il teorema di Reynolds si ottiene
$$\frac{D\mathbf{p}}{Dt}=\int_{\mathcal{P}_t}\rho\frac{D\mathbf{u}}{Dt}\,d\mathbf{x}$$
$$\frac{D\mathbf{L}}{Dt}=\int_{\mathcal{P}_t}\rho\mathbf{x}\times\frac{D\mathbf{u}}{Dt}\,d\mathbf{x}$$
Enunciamo ora la prima legge di Newton
\begin{theorem}
	Sia $\mathbf{f}(\mathcal{P},t)$ la risultante delle forze applicate alla porzione di corpo $\mathcal{P}$ al tempo $t$ allora
	$$\mathbf{f}((\mathcal{P},t))=\frac{D\mathbf{p}}{Dt}((\mathcal{P},t))$$
\end{theorem}
Si possono distinguere due tipi di forze, le forze di volume che sono applicate ad ogni punto del materiale, e le forze di contatto, responsabili dell'interazione tra sottoinsiemi del materiale.
Siano $\mathbf{b}(\mathbf{x},t)$ la densità di forze di volume e $\mathbf{t}(\mathbf{x},t,\mathbf{n})$ la densità di forze di contatto. Verrano considerati solo materiali di Cauchy in cui la densità di forza di contatto dipende solo dalla normale alla superficie e non dalla curvatura della stessa.
La forza applicata da una regione del corpo su di un'altra regione attraverso la superficie $\mathcal{S}$ è data da
$$\int_{\mathcal{S}}\mathbf{t}(\mathbf{n},\mathbf{x},t)\,da$$.
La prima legge di Newton per un materiale di Cauchy diventa
$$\mathbf{f}((\mathcal{P},t))=\frac{D\mathbf{p}}{Dt}((\mathcal{P},t)) = \int_{\partial\mathcal{P}_t}\mathbf{t}+\int_{\mathcal{P}_t}\mathbf{b}$$
Che è equivalente a chiedere che valga il principio dei lavori virtulali, cioè che per ogni $\mathcal{P}$, $t$ e per ogni $\mathbf{n}$ esista un tensore simmetrico $\boldsymbol{\sigma}$ detto tensore di stress di Cauchy tale che $\mathbf{t}(\mathbf{x},t,mathbf{n})=\boldsymbol{\sigma}(\mathbf{x},t)\mathbf{n}$ e
$$\int_{\mathcal{P}_t}\frac{D\mathbf{u}}{Dt}\cdot\mathbf{v}\,d\mathbf{x}+\int_{\mathcal{P}_t}\boldsymbol{\sigma}\boldsymbol{:}\Nabla\mathbf{v}\,d\mathbf{x}=\int_{\partial\mathcal{P}_t}\boldsymbol{\sigma}\mathbf{n}\cdot\mathbf{v}\,d\mathbf{x}+\int_{\mathcal{P}_t}\mathbf{b}\cdot\mathbf{v}\,d\mathbf{x}$$
per ogni funzione di test $\mathbf{v}$ sufficientemente regolare a supporto compatto in $\Omega$.
[$$\boldsymbol{\sigma}\boldsymbol{:}\Nabla\mathbf{v}=\sigma_{\alpha\beta}\frac{\partial v_{\alpha}}{\partial x_{\beta}}$$].
Se $\boldsymbol{\sigma}$ è sufficiente regolare allora vale la versione puntuale dell'equazione
$$\rho\frac{D\mathbf{u}}{Dt}=\Nabla\cdot\boldsymbol{\sigma}+\mathbf{b}$$.
Si può notare che in dimensione $d=2$ l'equazione fornisce $2$ condizioni per $2(\mathbf{u})+3(\boldsymbol{\sigma})$ incognite, mentre in dimensione $d=3$ l'equazione fornisce $3$ condizioni per $3(\mathbf{u})+6(\boldsymbol{\sigma})$.
I rimanenti gradi di libertà possono essere utilizzati per modellizzare vari tipi di materiali differenti, ad esempio i fluidi viscosi e i materiali elastici.

\subsection{Rotazioni e vorticità}

Sia $\mathbf{u} = (u,v,w)$ il campo delle velocità di un fluido allora il suo rotore 
\begin{equation*}
\boldsymbol{\xi} = \nabla \times \mathbf{u} = (\partial_y w - \partial_z v, \partial_z u - \partial_x w, \partial_x v - \partial_y u)
\end{equation*}
è detto campo della vorticità del fluido.

Si può dimostrare che $\mathbf{u}$ è localmente la somma di una traslazione rigida, di una deformazione e di una rotazione rigida di vettore $\frac{\boldsymbol{\xi}}{2}$. Cominciamo enunciando il seguente teorema che vale per ogni campo vettoriale definito su $\mathbb{R}^3$.

\begin{theorem}
Sia $\mathbf{x} \in \mathbb{R}^3$ e $\mathbf{y} = \mathbf{x} + \mathbf{h}$ un punto vicino. Allora 
\begin{equation}\label{eq:helmoltz}
\mathbf{u}(\mathbf{y}) = \mathbf{u}(\mathbf{x}) + \mathbf{D}(\mathbf{x}) \cdot \mathbf{h} + \frac{1}{2} \boldsymbol{\xi}(\mathbf{x}) \times \mathbf{h} + \mathcal{O} ( h^2)
\end{equation}
dove $\mathbf{D}$ è una matrice simmetrica.
\end{theorem}
\begin{proof}
Sia
\begin{equation}
\Nabla\mathbf{u}={\setlength\arraycolsep{2pt} 
\left(\begin{array}{ccc} 
\partial_x u&  \partial_y u &\partial_z u\\ 
\partial_x v&  \partial_y v &\partial_z v\\ 
\partial_x w&  \partial_y w &\partial_z w\\
\end{array}\right) 
}
\end{equation}
la matrice jacobiana di $\mathbf{u}$.
Per il teorema di Taylor
\begin{equation*}
\mathbf{u}(\mathbf{y}) = \mathbf{u}(\mathbf{x}) + \Nabla\mathbf{u}(\mathbf{x}) \cdot \mathbf{h} + \mathcal{O} ( h^2)
\end{equation*}
Siano rispettivamente
\begin{equation*}
\mathbf{D} = \frac{1}{2} [ \Nabla\mathbf{u} + \Nabla\mathbf{u} ^T]
\end{equation*}
la parte simmetrica e
\begin{equation*}
\mathbf{S} = \frac{1}{2} [ \Nabla\mathbf{u} - \Nabla\mathbf{u} ^T]
\end{equation*}
la parte antisimmetrica di $ \Nabla\mathbf{u}$.
\`E immediato verificare che $\mathbf{S}$ è nella forma
\begin{equation}
\mathbf{S}=\frac{1}{2}{\setlength\arraycolsep{2pt} 
\left(\begin{array}{ccc} 
0&  -\xi_3 &\xi_2\\ 
\xi_2&  0 &-\xi_1\\ 
-\xi_2&  \xi_1 &0\\
\end{array}\right) 
}
\end{equation}
e quindi
\begin{equation*}
\mathbf{S} \cdot \mathbf{h} = \frac{1}{2} \boldsymbol{\xi} \times \mathbf{h}
\end{equation*}
\qed
\end{proof}

La matrice $\mathbf{D}$ è detta tensore della deformazione. Dato che $\mathbf{D}$ simmetrica esiste una base ortonormale in cui è diagonale
\begin{equation}
\mathbf{D}={\setlength\arraycolsep{2pt} 
\left(\begin{array}{ccc} 
d_1 & 0 & 0 \\ 
0 & d_2 & 0 \\ 
0 & 0 & d_3 \\
\end{array}\right) 
}
\end{equation}.
Considerando solo il contributo di questo termine l'equazione \ref{eq:helmoltz} diventa 
\begin{equation*}
\frac{d h_i}{dt} = d_i h_i
\end{equation*}
Si tratta quindi di una contrazione o espansione nelle direzione della base ortonormale trovata.

Inoltre è possibile calcolare la variazione di volume
\begin{equation}
\begin{split}
\frac{d}{dt}h_1 h_2 h_3 &= \left [ \frac{dh_1}{dt} \right ] h_2h_3 + \left [ \frac{dh_2}{dt} \right ] h_1h_3 + \left [ \frac{dh_3}{dt} \right ] h_1h_2 \\
                                        &= \traccia \mathbf{D} h_1 h_2 h_3  \\
                                        &= \traccia  \frac{1}{2} [ \Nabla\mathbf{u} + \Nabla\mathbf{u} ^T] h_1 h_2 h_3 \\ 
                                        &= \Nabla \cdot \mathbf{u} h_1 h_2 h_3
\end{split}
\end{equation}
perché la traccia è invariante rispetto a cambi di coordinate. Si riottiene quindi che la deformazione è proporzionale a $ \Nabla \cdot \mathbf{u}$.

Il termine $\mathbf{u}(\mathbf{x})$ è una traslazione.

Il termine $\frac{1}{2}\boldsymbol{\xi}(\mathbf{x})\times\mathbf{h}$, usando lo stesso ragionamento usato nel caso di $\mathbf{D}$
\begin{equation*}
\frac{d\mathbf{h}}{dt}=\frac{1}{2}\boldsymbol{\xi}(\mathbf{x})\times\mathbf{h}
\end{equation*}
la cui soluzione è
\begin{equation*}
\mathbf{h}(t)=\mathbf{R}(t,\boldsymbol{\xi}(\mathbf{x}))\mathbf{h}(0)
\end{equation*}
con $\mathbf{R}$ rotazione rigida di angolo $t$ e asse $\boldsymbol{\xi}$. 

Le ipotesi fatte fino ad ora impediscono il presentarsi di qualsiasi forza tangenziale e quindi di qualsiasi meccanismo che possa causare o fermare una rotazione del fluido, cioè in ogni movimento del fluido la rotazione viene conservata. Data la stretta connessione tra rotazione e vorticità ci si aspetta che la vorticità sia connessa a questo fatto. 

[...][circolazioni,teorema di helmotz, teorema di kelvin]

\subsection{L'equazione di Navier Stokes}
Nei fluidi reali la quantità di moto viene diffusa anche grazie ai movimenti molecolari.
Rinunciamo ora all'ipotesi che il fluido sia ideale, e quindi ammettiamo anche forze di stress non in direzione normale.
Il teorema di Cauchy ci garantisce che la forza che agisce su $S$ è una d
$$F \, dA = -p(\mathbf{x},t) \mathbf{n} + \boldsymbol{\sigma}(\mathbf{x},t) \cdot \mathbf{n}$$
dove $\boldsymbol{\sigma}$ è il tensore degli stress.
Con questa nuova forza la legge di conservazione del momento diventa
$$\frac{d}{dt} \int_{\mathcal{W}_t} \rho \mathbf{u} \, dV = \int_{\partial \mathcal{W}_t} (p \mathbf{n} - \boldsymbol{\sigma} \cdot \mathbf{n} )\, dA$$
Faremo inoltre le seguenti ipotesi su $\boldsymbol{\sigma}$: \begin{itemize}
  \item $\boldsymbol{\sigma}$ dipende linearmente dal gradiente della velocità $\Nabla \mathbf{u}$;
  \item $\boldsymbol{\sigma}$ è invariante rispetto a rotazioni rigide, cioè data $\mathbf{U}$ matrice ortogonale
$$\boldsymbol{\sigma}(\mathbf{U} \cdot \Nabla \mathbf{u} \cdot \mathbf{U}^{-1}) = \mathbf{U} \cdot \boldsymbol{\sigma}( \Nabla \mathbf{u} ) \cdot \mathbf{U}^{-1}$$
\item $\boldsymbol{\sigma}$ è simmetrico.
\end{itemize} 

Dato che $\boldsymbol{\sigma}$ è simmetrico dipenderà solo dalla parte simmetrica di $\Nabla \mathbf{u}$, cioè dalla deformazione $\mathbf{D}$. Inoltre $\boldsymbol{\sigma}$ e $\mathbf{D}$ possono essere diagonalizzate simultaneamente in quanto commutano essendo $\boldsymbol{\sigma}$ una funzione lineare di $\mathbf{D}$.
Quindi gli autovalori di sigma sono funzioni lineari e simmetriche di quelli di D.
$$\sigma_i = \lambda (d_1 + d_2 + d_3) +2 \mu d_i$$
e quindi 
$$\boldsymbol{\sigma} = \lambda ( \Nabla\cdot \mathbf{u}) \mathbf{I} + 2 \mu \mathbf{D}
                                     = 2 \mu [ \mathbf{D} - \frac{1}{3} ( \Nabla\cdot \mathbf{u}) \mathbf{I} ] + \zeta ( \Nabla\cdot \mathbf{u})  \mathbf{I}
$$
$\mu$ è il primo coefficiente di viscosità e $\zeta$ è il secondo coefficiente di viscosità.

Usando il teorema del trasporto ora si ottiene l'equazione di Navier-Stokes

$$\rho \frac{D\mathbf{u}}{Dt} = -\nabla p + (\lambda+\mu) \nabla(\Nabla \cdot \mathbf{u}) + \mu \boldsymbol{\Delta} \mathbf{u}$$

Nel caso di fluidi incomprimibili le equazioni diventano
\begin{equation}
\boxed{
\begin{aligned}
\frac{D\mathbf{u}}{Dt} &= - \nabla p + \nu \boldsymbol{\Delta} \mathbf{u} \\
\Nabla \cdot  \mathbf{u} &= 0 \\
\mathbf{u} &= 0 \quad \text{su} \quad \partial \mathcal{D}
\end{aligned}
}
\end{equation}
L'ultima equazione è la cosiddetta condizione \emph{no-slip}, per cui il fluido è a riposo sulle pareti.

Nel caso dei fluidi incomprimibili quindi il tensore di stress $\boldsymbol{\sigma}$ dipende dal moltiplicatore di Lagrange relativo al vincolo di incomprimibilità $p$, anche detto pressione idrostatica, e dalla parte simmetrica di $\Nabla\mathbf{u}$, $\mathbf{D}=\frac{1}{2}(\Nabla\mathbf{u}+(\Nabla\mathbf{u})^{T})$
$$\boldsymbol{\sigma}=-p\mathbf{I}+\eta(\Nabla\mathbf{u}+(\Nabla\mathbf{u})^{T})$$
la cui divergenza è
$$\Nabla\cdot\boldsymbol{\sigma}=-\nabla p+\eta\Delta\mathbf{u}$$.
[numero di Reynolds][metodi di proiezione per la pressione e equazione di Stokes]

\subsection{Le equazioni della fluidodinamica in forma debole}
%The Finite Element Immersed Boundary Method, L. Heltai ,Ph.D. Thesis , 2006, capitolo 1 19-35
Scriviamo ora l'equazione di Navier-Stokes in forma variazionale.

Sia
\begin{equation*}
\mathbf{V} = H^1_0(\Omega)^d = \bigg \{ \mathbf{v} \in L^2(\Omega)^d \quad \tc \quad \Nabla \mathbf{v} \in L^2(\Omega)^{d \times d} \quad \text{e} \quad \mathbf{v}\mid_{\partial \Omega} = 0 \bigg \}
\end{equation*}
lo spazio delle velocità e 
\begin{equation*}
Q = L^2_0(\Omega) = \left  \{ q \in L^2(\Omega) \quad \tc \quad \int_{\Omega}q=0 \right  \}
\end{equation*}
lo spazio delle pressioni. Indichiamo con $\mathbf{V}'$ lo spazio duale di $\mathbf{V}$.

Dati $\mathbf{b} \in \mathbf{V}'$ e $\mathbf{u}_0 \in \mathbf{V}$, per ogni $t \in (0,T)$ trovare $(\mathbf{u},p) \in \mathbf{V} \times Q$ tale che
\begin{align*}
\rho \left ( \frac{d}{dt}\ps{\mathbf{u}(t)}{\mathbf{v}} + \ps{\mathbf{u}(t)\cdot \Nabla\mathbf{u}(t)}{\mathbf{v})}\right)+ \nu\ps{\Nabla\mathbf{u}(t)}{\Nabla\mathbf{v}}-\ps{\Nabla \mathbf{v}}{p(t)}=\ps{\mathbf{b}}{\mathbf{v}} \qquad & \forall \mathbf{v}\in\mathbf{V} \\
\ps{\Nabla\cdot\mathbf{u}(t)}{q}=0 \qquad & \forall q\in\mathbf{Q} \\
\mathbf{u}(\mathbf{x},0)=\mathbf{u}_0(\mathbf{x}) \qquad & \forall \mathbf{x}\in\Omega
\end{align*}
dove $(\cdot,\cdot)$ rappresenta il prodotto di dualità tra $\mathbf{V}$ e $\mathbf{V}'$ e il prodotto interno di $Q$ a seconda degli argomenti coinvolti.

Definendo i seguenti operatori associati ai termini dell'equazione di Navier-Stokes
\begin{align*}
(M\mathbf{u})(\mathbf{v})&=\rho\ps{\mathbf{u}}{\mathbf{v}} \quad &\mathbf{u},\mathbf{v}\in \mathbf{V}\\
(A\mathbf{u})(\mathbf{v})&=a(\mathbf{u},\mathbf{v})=\eta\ps{\Nabla\mathbf{u}}{\Nabla\mathbf{v}} \quad &\mathbf{u},\mathbf{v}\in \mathbf{V}\\
(N(\mathbf{w})\mathbf{u})(\mathbf{v})&=c(\mathbf{w},\mathbf{u},\mathbf{v})=\rho\ps{\mathbf{w}\cdot\Nabla\mathbf{u}}{\Nabla\mathbf{v}} \quad &\mathbf{w},\mathbf{u},\mathbf{v}\in \mathbf{V}\\
(B\mathbf{u})(q)&=b(\mathbf{u},q)=\ps{\Nabla\cdot\mathbf{u}}{q} \quad &\mathbf{u}\in \mathbf{V},q\in Q
\end{align*}
possiamo riscrivere le equazioni come
\begin{align*}
&\rho\frac{d}{dt}\ps{\mathbf{u}}{\mathbf{v}}+a(\mathbf{u},\mathbf{v})+c(\mathbf{w},\mathbf{u},\mathbf{v})+b(\mathbf{v},p)=\ps{\mathbf{b}}{\mathbf{u}} \quad &\forall\mathbf{v}\in\mathbf{V}\\
&b(\mathbf{u},q)=0 \quad &\forall q\in Q\\
&\mathbf{u}(\mathbf{x},0)=\mathbf{u}_0(\mathbf{x})&
\end{align*}
oppure, in termini di operatori degli spazi duali, come
\begin{align*}
&M\frac{\partial\mathbf{u}}{\partial t}+A\mathbf{u}+N(\mathbf{u})\mathbf{u}+B^{T}p=\mathbf{b}\\
&B\mathbf{u}=0\\
&\mathbf{u}(\mathbf{x},0)=\mathbf{u}_0(\mathbf{x})&
\end{align*}

\paragraph{Stima dell'energia}
\begin{theorem}
Per ogni $t \in (0,T)$ sia $(\mathbf{u}(t),p(t)) \in \mathbf{V} \times Q$ la soluzione del problema con $\mathbf{b}=0$, allora
\begin{equation*}
\rho \frac{1}{2} \frac{d}{dt} \Vert \mathbf{u} (t) \Vert^2_{0,\Omega} + \eta \Vert \Nabla \mathbf{u} (t) \Vert^2_{0,\Omega} = 0
\end{equation*}
\end{theorem}
\begin{proof}
Per dimostrarlo si prendano $\mathbf{v}=\mathbf{u}$ e $q=p$ come funzioni di test nelle equazioni, allora $b(\mathbf{u},p)=0$ e $c(\mathbf{u},\mathbf{u},\mathbf{u})=0$ e quindi
\begin{equation*}
\rho\ps{\frac{\partial \mathbf{u}}{\partial t}}{\mathbf{u}}+\eta\ps{\Nabla\mathbf{u}}{\Nabla\mathbf{u}} = 
\rho\frac{1}{2}\frac{\partial} {\partial t}\Vert\mathbf{u}\Vert^2+\eta\Vert \Nabla\mathbf{u}\Vert^2 = 0
\end{equation*}
\end{proof}
Il teorema afferma che la variazione di energia cinetica è pari all'energia dissipata dai fenomeni viscosi e che l'energia totale del fluido è limitata dall'energia cinetica iniziale.
I fluidi governati dalle equazioni di Navier-Stokes incomprimibili non possono accumulare energia se non sottoforma di energia cinetica dato che il fluido è incomprimibile e lo stress $\boldsymbol{\sigma}$ non dipende dalla deformazione del fluido. 

\section{Teoria matematica dell'elasticità}
%Marsden Mathematical Foundation of Elasticity
La caratteristica dei materiali elastici, a differenza dei fluidi, è quella di tendera a ritornare nella proprio posizione di equilibrio, cioè la configurazione che minimizza una data funzione energia dipendente dalla configurazione e specifica per il materiale, detta configurazione di riferimento o configurazione di equilibrio, che rappresenta lo stato non deformato del materiale. Questa posizione e la traiettoria seguita dai punti del corpo per raggiungerla dipende dalla natura del materiale elastico e in definitiva dalla sua legge costitutiva.

Anche per i materiali elastici valgono le leggi di conservazione della massa e del momento, e quindi dati la densità del materiale $\rho$, il tensore degli stress di Cauchy $\boldsymbol{\sigma}$ e la forza esterna $\mathbf{b}$ sarà ancora valida l'equazione del moto di Cauchy

\begin{equation*}
\rho\frac{d\mathbf{u}}{dt} = \Nabla\cdot \boldsymbol{\sigma} + \rho\mathbf{b}
\end{equation*}

\subparagraph{In coordinate lagrangiane}
\'E possibile esprimere il tensore di Cauchy in coordinate lagrangiane sfruttando le quantità materiali al posto di quelle spaziali, ottenendo così i vettori degli stress di Piola-Kirchhoff.
Il primo vettore di stress di Piola-Kirchhoff $\mathbf{T}(\mathbf{s},t,\mathbf{n})$ è un vettore parallelo al vettore di stress di Cauchy $\mathbf{t}(\mathbf{x},t,\mathbf{n})$, ma misura la forza per unità di area non deformata,
cioè
\begin{equation*}\mathbf{T}\,d\mathbf{A} = \mathbf{t}\,da
\end{equation*}

Sfruttando l'identità di Piola
\begin{equation*}
\Nabla \cdot (J\mathbf{F}^{-1}) = 0
\end{equation*}
cioè
\begin{equation*}
\frac{\partial}{\partial x_j}(J\frac{\partial X_j}{\partial s_i}) = 0
\end{equation*}
che si ottiene osservando che $J\mathbf{F}^{-1}$ è la matrice dei cofattori di $\mathbf{F}$ cioè
\begin{equation*}
(J\mathbf{F}^{-1})^r_k = \epsilon_{ijk}\epsilon^{pqr}\mathbf{F}^i_p\mathbf{F}^j_q
\end{equation*}
dove $\epsilon$ è il tensore di Levi-Civita possiamo definire il primo tensore di Piola-Kirchhoff come 
\begin{equation*}
\mathbf{P}=J\boldsymbol{\sigma}\mathbf{F}^{-T}
\end{equation*}
allora l'equazione di Cauchy diventa
\begin{equation*}
\frac{d}{dt}\int_{\mathcal{U}}\rho_{\mathcal{B}}\mathbf{V}\, d\mathbf{X} = \int_\mathcal{U}\Nabla\cdot\mathbf{P}+\rho_{\mathcal{B}}\mathbf{B}\, d\mathbf{X}
\end{equation*}

Definiamo il secondo tensore di Piola-Kirchhoff come
\begin{equation*}
\mathbf{S}=\mathbf{F}^{-1}\mathbf{P}
\end{equation*}
cioè
\begin{equation*}
S_{ij} = \sum \frac{\partial X_i}{\partial s_k}P_{kj}
\end{equation*}


\paragraph{Materiali elastici}
Per risolvere le equazioni del moto bisogna determinare come la forza nella seconda equazione di Newton dipende dalla posizione e dalla velocità delle particelle che costituiscono il corpo e come lo stress dipende dal moto nell'equazione del moto di Cauchy. 
Un materiale è elastico se esiste una funzione detta funzione costitutiva $\widehat{\mathbf{P}}$ tale che
\begin{equation*}
\mathbf{P}(\mathbf{X},t)=\widehat{\mathbf{P}}(\mathbf{X},\mathbf{F}(\mathbf{X},t))
\end{equation*}
cioè il primo tensore di Piola-Kirchhoff dipende dalla deformazione ma non dalle sue derivate, ad esempio dalla velocità come nel caso dei fluidi e dei corpi viscosi.
\paragraph{Materiali iperelastici}
Un materiale si dice iperelastico se esiste una funzione detta funzione energia $\widehat{\mathbf{W}}$ tale che 
\begin{equation*}
\widehat{\mathbf{P}}(\mathbf{X},\mathbf{F}(\mathbf{X},t))=\rho_{\mathcal{B}}\frac{\partial \widehat{\mathbf{W}}}{\partial \mathbf{F}}
\end{equation*}
\subparagraph{\emph{Indipendenza dal sistema di riferimento}}
Le leggi costitutive del corpo in esame non devono dipendere dal particolare sistema di riferimento usato per descriverlo.
In particolare sia $\mathbf{Q}$ una matrice ortogonale, allora 
$$\mathbf{T}(\mathbf{X},\mathbf{F},\mathbf{N}) = \widehat{\mathbf{P}}(\mathbf{X},\mathbf{F})\mathbf{N} $$
$$\widehat{\mathbf{P}}(\mathbf{X},\mathbf{Q}\mathbf{F}) = \mathbf{Q}\widehat{\mathbf{P}}(\mathbf{X},\mathbf{F}) $$
Per un materiale iperelastico 
$$\widehat{\mathbf{W}}(\mathbf{X},\mathbf{Q}\mathbf{F}) = \widehat{\mathbf{W}}(\mathbf{X},\mathbf{F}) $$
e quindi
$$\widehat{\mathbf{W}}(\mathbf{X},\mathbf{F}) = \widehat{\mathbf{W}}(\mathbf{X},\sqrt{\mathbf{C}}) $$
L'indipendenza dal sistema di riferimento implica la conservazione momento angolare e la simmetria dei tensori di stress.
Osserviamo inoltre per i materiali iperelastici vale
\begin{equation*}
\widehat{\mathbf{S}} = 2\rho_{\mathcal{B}}\frac{\partial\widehat{W}}{\partial\mathbf{C}}
\end{equation*}
\paragraph{Materiali isotropici}
Un materiale è omogeneo se $\widehat{\mathbf{P}}$ non dipende esplicitamente da $\mathbf{X}$, cioè il comportamento del materiale non dipende dal punto del materiale in esame ma solo dalla sua deformazione.

Un materiale è isotropico se
$$\widehat{\mathbf{P}}(\mathbf{X},\mathbf{Q}\mathbf{F}) = \widehat{\mathbf{P}}(\mathbf{X},\mathbf{F}) $$
per ogni matrice ortogonale $\mathbf{Q}$.
Inoltre se un materiale è iperelastico vale 
$$\widehat{\mathbf{W}}(\mathbf{X},\mathbf{F}\mathbf{Q}) = \widehat{\mathbf{W}}(\mathbf{X},\mathbf{F}) $$

Un materiale è iperelastico, indipendente dal sistema di riferimento, omogeneo e isotropico se e solo se
$$\widehat{\mathbf{W}}(\mathbf{F}) = \Phi(\lambda_1,\lambda_2,\lambda_3) $$
con $\Phi$ funzione simmetrica degli sforzi principali $\lambda_1$, $\lambda_2$ e $\lambda_3$.
Si può mostrare inoltre che
$$\mathbf{S}=\alpha_0\mathbf{I}+\alpha_1\mathbf{C}+\alpha_2\mathbf{C}^2 $$
e
$$\boldsymbol{\sigma}=\beta_0\mathbf{I}+\beta_1\mathbf{b}+\beta_2\mathbf{b}^2 $$
con $\alpha_i$ e $\beta_i$ funzioni scalari degli invarianti di $\mathbf{C}$  e di $\mathbf{b}$.
\paragraph{Elasticità lineare}
Quando il corpo in esame è soggetto solo a piccoli spostamenti e a piccole deformazioni è possibile ottenere una teoria semplificata linearizzando le equazioni non lineari del moto.

Sia $\widehat{\mathbf{P}}$ la funzione costitutiva di un materiale elastico omogeneo non lineare tale che $\widehat{\mathbf{P}}(\mathbf{I})=0$, cioè la configurazione di riferimento ha stress nullo.
Sia $\phi_\epsilon(\mathbf{X},t)$ una famiglia di moti dipendenti dal parametro $\epsilon \ll1$ tale che $\phi_0(\mathbf{X},t) = \mathbf{X}$. Quindi $\phi_0$ soddisfa le equazioni del moto
$$\rho_\mathcal{B}\frac{\partial \mathbf{V}}{\partial t} = \Nabla\cdot\widehat{\mathbf{P}}(\mathbf{F}) $$.
Si supponga che anche $\phi_\epsilon$ soddisfi la stessa equazione per ogni $\epsilon$.
Sia 
$$\phi_\epsilon(\mathbf{X},t) = \mathbf{X}+\epsilon\mathbf{u}(\mathbf{X},t) + \mathcal{O}(\epsilon^2)$$
l'espansione in serie di $\phi_\epsilon$.
Il problema di riduce a trovare $\mathbf{u}$.
Osservato che 
$$\mathbf{u}=\frac{\partial}{\partial \epsilon}\phi_\epsilon \big|_{\epsilon=0}$$
e
$$\frac{\partial}{\partial t}\mathbf{u}=\frac{\partial}{\partial \epsilon}\mathbf{V}_\epsilon \big|_{\epsilon=0}$$
differenziando l'equazione (?.?) si ottiene
$$\rho_{\mathcal{B}} \frac{\partial}{\partial \epsilon}\mathbf{V}_\epsilon = \Nabla\cdot\widehat{\mathbf{P}}(\mathbf{F_\epsilon)}$$
che in $\epsilon=0$ vale
$$\rho_{\mathcal{B}} \frac{\partial^2\mathbf{u}^i}{\partial t^2} = \frac{\partial}{\partial \mathbf{X}^j}(\frac{\partial\widehat{\mathbf{P}}^{ij}}{\partial\mathbf{F}^k_l}(\mathbf{I})\frac{\partial\mathbf{u}^k}{\partial\mathbf{X}^l})$$
Definiamo il tensore classico di elasticità come
$$ c_{ijkl}(\mathbf{X})= \frac{\partial\widehat{\mathbf{P}}^{ij}}{\partial\mathbf{F}^k_l}(\mathbf{I})$$.
Se il materiale è omogeneo allora i valori di $c_{ijkl}$ sono costanti in tutto il corpo e l'equazione dell'elasticità lineare diventa
$$\rho_{\mathcal{B}} \frac{\partial^2\mathbf{u}^i}{\partial t^2} = \frac{\partial^2\mathbf{u}^k}{\partial\mathbf{X}^j\partial\mathbf{X}^l}$$.
Se il materiale è omogeneo e isotropico allora esistono due costani $\lambda$ e $\mu$ dette moduli di Lamè per cui
$$c_{ijkl}=\lambda\delta_{ij}\delta_{kl}+\mu(\delta_{ik}\delta_{jl}+\delta_{il}\delta_{jk})$$.
\paragraph{Un fluido elastico}
Si consideri un corpo non lineare iperelastico per cui
$$\widehat{W}(\mathbf{F}) = h(|\mathbf{F}|)$$.
Allora è omogeneo, indipendente dal sistema di riferimento e isotropico e il primo tensore di Piola-Kirchhoff vale
$$\widehat{\mathbf{P}}=\rho_{\mathcal{B}}\frac{\partial \widehat{W}}{\partial \mathbf{F}} = \rho_{\mathcal{B}}h'(J)\frac{\partial \mathbf{J}}{\partial \mathbf{F}}$$
con $J = |\mathbf{F}|$.
Osservato che 
$$\frac{\partial J}{\partial \mathbf{F}} = J\mathbf{F}^{-T}$$
allora il tensore di stress di Cauchy vale
$$\boldsymbol{\sigma}=(\frac{1}{J})\widehat{\mathbf{P}}\mathbf{F}^{T}=\rho_{\mathcal{B}}h'(J)\mathbf{I}$$.
Dato che
$$p(\rho)=-\rho_{\mathcal{B}}h'(J)$$
allora
$$\boldsymbol{\sigma}=-p(\rho)\mathbf{I}$$.
L'equazione del moto di Cauchy diventa
$$\rho\frac{D \mathbf{u}}{Dt} = -\Nabla p+\rho\mathbf{b}$$
che è l'equazione di Eulero per i fluidi perfetti comprimibili, che quindi sono un sottoinsieme dei materiali elastici.
Nei fluidi viscosi invece $\boldsymbol{\sigma}$ dipende non solo da $\mathbf{F}$ ma anche da $\Nabla\mathbf{v}$.
[distinguere fluidi da solidi mediante gruppo di simmetria]
\paragraph{Incomprimibilità}
Un materiale è incomprimibile se $J\equiv 1$ o $\Nabla\cdot\mathbf{v}=0$.
Aggiungendo all'espressione del primo tensore di Piola-Kirkhhoff moltiplicatore di Lagrange associato al vincolo di incomprimibilità si ottiene
\begin{equation*}
\mathbf{P}(\mathbf{X},t)=-p\mathbf{F}^{-T}+\widehat{\mathbf{P}}(\mathbf{X},\mathbf{F}(\mathbf{X},t))
\end{equation*}

Per $\widehat{\mathbf{P}}=0$  si ottiene l'equazione di Eulero per i fluidi perfetti incomprimibili
\begin{align*}
&\rho\frac{D \mathbf{v}}{Dt} = -\Nabla p+\rho\mathbf{b} \\
&\Nabla\cdot\mathbf{v}=0
\end{align*}

\paragraph{Materiali di Mooney-Rivlin}
Un esempio solido eleastico non lineare è la gomma. Questa si comporta come un solido incompressibile, omogeneo, iperelastico e isotropico.
La funzione che descrive l'energia accumulata da questo tipo di materiali è dovuta a Mooney (1940) e Rivlin (1948)
\begin{multline*}
\Phi(\lambda_1,\lambda_2,\lambda_3) = \alpha(\lambda_1^2+\lambda_2^2+\lambda_3^2-3) + \\ \beta((\lambda_2\lambda_3)^2+(\lambda_3\lambda_1)^2+(\lambda_1\lambda_2)^2 -3)
\end{multline*}
dove $\lambda_1,\lambda_2,\lambda_3$ sono gli stress principali, $\alpha$ e $\beta$ sono due costanti positive. 
Si osservi che 
\begin{equation*}
\Phi(1,1,1) = 0
\end{equation*}
Un materiale si dice neo-Hookeano se $\beta=0$.
Per i materiali comprimibili è possibile usare il modello di Hadamard
\begin{equation*}
\Phi(\lambda_1,\lambda_2,\lambda_3)\alpha(\lambda_1^2+\lambda_2^2+\lambda_3^2-3) + h(\lambda_1,\lambda_2,\lambda_3)
\end{equation*}
con $h(\delta)$ tale che
\begin{align*}
&h(\delta) \quad\text{ha un minimo in}\quad\delta=1 \\
&h(\delta)\to\infty\qquad\delta\to 0\\
&h(\delta)\to\infty\qquad\delta\to\infty\\
\end{align*}

In generale l'equazione costitutiva del materiale in esame deve essere desunta attraverso misurazioni empiriche.

%Heltai

\section{Teoria dell'elasticità strikes back}

Un corpo si dice elastico se tende a ritornare nella sua configurazione originale ( o di equilibrio ) dopo essere stato deformato. Le particelle di un corpo elastico tendono a recuperare la loro posizione relativa rispetto alle altre particelle, mentre in un fluido il comportamento è stabilito solo dalla loro velocità relativa ma non dalla loro posizione. I corpi elastici pur rispettando le stesse leggi di conservazione dei fluidi presentano diverse leggi costitutive che hanno l'effetto di mantenere invariata la posizione relativa delle particelle, per questo motivo le leggi costitutive dei materiali elastici sono date prevalentemente nel riferimento lagrangiano.

\paragraph{Materiali iperelastici}

Un materiale incomprimibile si dice iperelastico se è possibile definire una funzione di densità di energia potenziale

\begin{equation*}
W(\mathbb{F},\mathbf{s}) > 0
\end{equation*}

associata alla deformazione $\mathbb{F}$.

Il comportamento di un materiale iperelastico dipende solo dalla sua configurazione attuale e non presenta fenomeni di isteresi.

Sia 
\begin{equation*}
E [  \mathbf{X}(t) ] = \int_{\omega} W( \mathbb{F}, \mathbf{s} )) \, d\mathbf{s}
\end{equation*}
l'energia potenziale totale del corpo.

Grazie ad alcune considerazione fisiche possiamo caratterizzare la funzione $W$:

1. In assenza di deformazione non c'è energia accumulata nel corpo $W(\mathbb{I}) = 0$

2. la densità di energia diverge a $+\infty$ quando il corpo viene compresso o espanso indefinitamente  

\begin{equation*}
\vert \mathbb{F} \vert \rightarrow 0 \Rightarrow W(\mathbb{F}) \rightarrow +\infty
\end{equation*}

\begin{equation*}
\vert \mathbb{F} \vert \rightarrow +\infty \Rightarrow W(\mathbb{F}) \rightarrow +\infty
\end{equation*}

3.  la densità di energia non varia in seguito a moti rigidi del corpo, cioè se $\mathbb{F}^*$ è ottenuta con un moto rigido da $\mathbb{F}$ allora 
\begin{equation*}
W(\mathbb{F}^*) = W(\mathbb{F})
\end{equation*}

Ricordando che $\mathbb{F}$ non cambia in seguito a traslazioni rigide, quindi possiamo scrivere

\begin{equation*}
\mathbb{F}^* = R\mathbb{F}
\end{equation*}
con $R$ rotazione tale che 
\begin{align*}
&RR^T = \mathbb{I} \\
&\Vert R \Vert > 0
\end{align*}

Ogni gradiente di deformazione $\mathbb{F}$ può essere decomposto nel prodotto tra una rotazione e il tensore simmetrico degli stress $\mathbb{U}$. Quindi

\begin{equation*}
W(\mathbb{F}) = W(R\mathbb{U}) = W(\mathbb{U})
\end{equation*}


Possiamo ora definire il tensore di Cauchy-Green e il tensore di Green
\begin{equation*}
\mathbb{C} = \mathbb{F}^T\mathbb{F} = \mathbb{U}^T\mathbb{U}
\end{equation*}
\begin{equation*}
\mathbb{G} = \frac{1}{2}(\mathbb{C}-\mathbb{I})
\end{equation*}

che sono indipendenti da moti rigidi del corpo nel riferimento euleriano. Se inoltre il materiale è isotropico allora $W$ non dipende da rotazioni del corpo nel riferimento lagrangiano e può essere espresso come funzione degli invarianti di $\mathbb{C}$ ( o $\mathbb{G}$ )
\begin{align*}
&I_C = \traccia(\mathbb{C}) \\
&II_C = \traccia(\mathbb{C}^2) \\
&III_C = \det(\mathbb{C}) \\
\end{align*}

Esprimere il tensore degli stress di Cauchy $\boldsymbol(\sigma)$ in termini di $W$ a causa del fatto che sono definiti in due sistemi di riferimento differenti è in genere molto complicato.
Per calcolare lo sforzo generato nel materiale è conveniente quindi introdurre i tensori di Piola-Kirchhoff, che ne sono la riscrittura in termini di coordinate lagrangiane

Definiamo il primo tensore degli stress di Piola-Kirchhoff come
\begin{equation*}
(\mathbb{P}(\mathbf{s},t))_{\alpha,\beta} = \frac{\partial W}{\partial \mathbb{F}_{\alpha,\beta}} (\mathbf{s},t) =\left( \frac{\partial W}{\partial \mathbb{F}} (\mathbf{s},t)\right)_{\alpha,\beta}
\end{equation*}
e il secondo tensore degli stress di Piola-Kirchhoff come
\begin{equation*}
\mathbb{S}(\mathbf{s},t)=\frac{\partial W}{\partial\mathbb{G}}(\mathbf{s},t)=2\frac{\partial W}{\partial\mathbb{C}}(\mathbf{s},t)
\end{equation*}
Osserviamo che
\begin{equation*}
\mathbb{P}=\mathbb{F}\mathbb{S}
\end{equation*}
e che in genere $\mathbb{P}$ non è un tensore simmetrico mentre $\mathbb{S}$ è sempre un tensore simmetrico.

Il primo tensore di Piola-Kirchhoff rappresenta la forza per unità di area indeformata generata del materiale a causa della deformazione, il secondo tensore di Piola-Kirchhoff non ha invece una interpretazione fisica diretta.

Possiamo scrivere la relazione tra il tensore di Cauchy il primo tensore di Piola-Kirchhoff come
\begin{equation*}
\int_{\partial\mathcal{P}_t} \boldsymbol{\sigma}\mathbf{n} \, da = \int_{\partial\mathcal{P}} \mathbb{P}\mathbf{N} \, dA
\end{equation*}
la relazione puntuale tra i due è data da
\begin{equation*}
\mathbb{P}(\mathbf{s},t) = \vert \mathbb{F}(\mathbf{s},t) \vert \mathbf{\sigma}(\mathbf{X}(\mathbf{s},t),t) \mathbb{F}^{-T}(\mathbf{s},t)
\end{equation*}

Possiamo scrivere la legge di conservazione del momento in termini del primo tensore di Piola-Kirhhoff
\begin{equation*}
M\frac{\partial^2\mathbf{X}}{\partial t^2} = \Nabla_\mathbf{s} \cdot \mathbb{P} + \mathbf{B}
\end{equation*}
con 
\begin{equation*}
(\Nabla_\mathbf{s} \cdot \mathbb{P} )_{\alpha} = \sum_{\beta}\frac{\partial\mathbb{P}_{\alpha,\beta}}{\partial s_\beta}
\end{equation*}
e
\begin{equation*}
\mathbf{B}(\mathbf{s},t)=\vert\mathbb{F}(\mathbf{s},t)\vert\mathbf{b}(\mathbf{X}(\mathbf{s},t))
\end{equation*}

\paragraph{Incomprimibilità}

Il vincolo di incomprimibilità nel riferimento lagrangiano si può esprimere come

\begin{equation*}
\frac{d}{dt}\int_{\mathcal{P}_t}\,d\mathbf{x} = \int_{\mathcal{P}}\frac{\partial}{\partial t}\vert\mathbb{F}\vert\,d\mathbf{s} = 0
\end{equation*}
che equivale a chiedere che $\vert\mathbb{F}\vert$ rimanga costante nel tempo, e quindi senza perdità di generalità si può assumere che 
\begin{equation*}
\vert \mathbb{F} \vert = 1
\end{equation*}

L'aggiunta del vincolo di incomprimibilità rende necessario correggere la funzione densità di energia

\begin{equation*}
W^{\text{inc}}(\mathbb{F}) = W(\mathbb{F})-P(\vert\mathbb{F}\vert - 1)
\end{equation*}
dove $P$ è il moltiplicatore di Lagrange associato alla condizione $\mathbb{F}\vert=1$.

Possiamo così calcolare il primo tensore di Piola-Kirchhoff
\begin{equation*}
\mathbb{P}^{\text{inc}}=\frac{\partial W}{\partial t}-\vert\mathbb{F}\vert P\mathbb{F}^{-t}
\end{equation*}

$P$ rappresenta l'equivalente della pressione idrostatica $p$ presente in coordinate euleriane nelle equazioni di Navier-Stokes.

Un esempio di corpo elastico è rappresentato dal modello neo-Hookiano in cui la legge costitutiva è data da
\begin{equation*}
W = \frac{1}{2}\mu(\traccia(\mathbb{C})-d)-\mu\ln(\vert\mathbb{F}\vert)+\frac{1}{2}\lambda\ln(\vert\mathbb{F}\vert)^2
\end{equation*}
Usando questa legge costituitiva il primo tensore di Piola-Kirchhoff è dato da
\begin{equation*}
\mathbb{P} = \mu(\mathbb{F}-\mathbb{F}^{-T})+\lambda\ln(\vert\mathbb{F}\vert)\mathbb{F}^{-T}
\end{equation*}
Osserviamo che nel caso di materiali incomprimibili ( per cui $\ln(\vert\mathbb{F}\vert)=0$ ) $\mathbb{P}$ è lineare.

\subsection{Equazioni in forma forte}

Sia $\mathcal{B} \subseteq \mathbb{R}^3$ una regione poligonale convessa sulla cui superficie $\partial\mathcal{B}$ è applicata una forza la cui densità di forza per superficie è indicata con $\mathbf{T}$.

Dati $\mathbf{X}_0$, $\mathbf{X}^{'}_0$ e $\mathbf{T}$ trovare $\mathbf{X}$ e $P$ tali che:
\begin{align*}
&M\frac{\partial^2\mathbf{X}}{\partial t^2} = \Nabla_\mathbf{s} \cdot \mathbb{P} &\qquad \forall(\mathbf{s},t)\in\mathcal{B}\times(0,T)\\
&\mathbb{P} = \frac{\partial W}{\partial\mathbb{F}} - \vert\mathbb{F}\vert P\mathbb{F}^{-T} + \eta\vert\mathbb{F}\vert \frac{\partial}{\partial t}(\mathbb{F}\mathbb{F}^{-T} + (\mathbb{F}\mathbb{F}^{-T})^T)\mathbb{F}^{-T}&\qquad \forall(\mathbf{s},t)\in\mathcal{B}\times(0,T)\\\
&\vert\mathbb{F}\vert =1&\qquad \forall(\mathbf{s},t)\in\mathcal{B}\times(0,T)\\\
&\mathbf{X}(\mathbf{s},0)=\mathbf{X}_0(s)&\qquad \forall\mathbf{s}\in\mathcal{B}\\\
&\frac{\partial \mathbf{X}}{\partial t}(\mathbf{s},0)=\mathbf{X}^{'}_0(\mathbf{s})&\qquad \forall\mathbf{s}\in\mathcal{B}\\
&\mathbb{P}\mathbf{N}=-\mathbf{T}&\qquad \forall(\mathbf{s},t)\in\partial\mathcal{B}\times(0,T)
\end{align*}

$\mathbf{T}$ rappresenta la trazione esterna applicata sulla superficie di $\mathcal{B}$, compensata dalla reazione del materiale.

Il problema può essere semplificato trattando la costante di incomprimibilità attraverso metodi di penalizzazione e linearizzando il termine viscoso o trattandolo nel riferimento euleriano.

\subsection{Equazioni in forma debole}

Sia
\begin{equation*}
S = \{ \mathbf{X} \in H^1(\mathcal{B})^d \cap W^{1,\infty}(\mathcal{B})^d \quad \texttt{t.c.} \quad \vert \boldsymbol {\nabla}_{\mathbf{s}}\mathbf{X}\vert = \vert \mathbb{F} \vert > 0, \mathbf{X}(\mathcal{B}) \subseteq \Omega \}
\end{equation*}
il sottospazio di $H^1(\mathcal{B}^d)$ che contiene le sole configurazioni ammesse del materiale e
\begin{equation*}
\mathbf{P} = L^2(\mathcal{B})^{d\times d}
\end{equation*}
lo spazio dei gradienti di $H^1(\mathcal{B}^d)$.

Dati $\mathbf{T} \in H^{-\frac{1}{2}}(\partial\mathcal{B})^d$ e $\mathbf{X}_0,\mathbf{X}^{'}_0 \in S$ per ogni $t \in (0,t)$ trovare $\mathbf{X}(t) \in S$ tale che 
\begin{align*}
&\frac{d^2}{dt^2}\ps{\mathbf{X}(t)}{\mathbf{Y}} + \ps{\mathbb{P}(\mathbb{F}(t))}{\Nabla_\mathbf{s} \mathbf{Y}} = -\int_{\partial\mathcal{B}}\mathbf{T}(t)\cdot\mathbf{Y}\,dA\qquad&\forall\mathbf{Y}\in H^1(\mathcal{B}^d)\\
&\mathbf{X}(\mathbf{s},0)=\mathbf{X}_0(\mathbf{s})\qquad&\forall\mathbf{s}\in\mathcal{B}\\
&\frac{\partial\mathbf{X}}{\partial t}(\mathbf{s},0)=\mathbf{X}^{'}_0(\mathbf{s})&\forall\mathbf{s}\in\mathcal{B}
\end{align*}

Le equazioni possono essere scritte in termini di operatori come fatto per le equazioni di Navier-Stokes.
Definiamo i seguenti operatori
\begin{align*}
&\ps{M\mathbf{X}}{\mathbf{Y}}=\ps{\mathbf{X}}{\mathbf{Y}}\qquad&\forall\mathbf{X},\mathbf{Y}\in H^1(\mathcal{B}^d) \\
&\ps{D\mathbf{Y}}{\mathbb{S}}=\ps{D^T\mathbb{S}}{\mathbf{Y}}=\ps{\mathbb{S}}{\Nabla_{\mathbf{s}}\mathbf{Y}}\qquad&\forall\mathbf{Y}\in H^1(\mathcal{B}^d),\mathbb{S}\in\mathbb{P}\\
&\ps{\gamma_{\mathcal{B}}\mathbf{T}}{\mathbf{Y}}=\int_{\partial \mathcal{B}}\mathbf{T}\cdot\mathbf{Y}\,d\mathbf{A}\qquad&\forall\mathbf{T}\in L^2(\mathcal{B})^d,\mathbf{Y}\in H^1(\mathcal{B}^d) \\
\end{align*}
allora il problema può essere riscritto come
\begin{align*}
&M\frac{\partial^2 \mathbf{X}}{\partial t^2}(t)+D^T\mathbb{P}(\mathbb{F}(t))=-\gamma_{\mathcal{B}}\mathbf{T}
\qquad&H^1(\mathcal{B}^d) \\
&\mathbf{X}(\mathbf{s},0)=\mathbf{X}_0(\mathbf{s})\qquad&\forall\mathbf{s}\in\mathcal{B}\\
&\frac{\partial\mathbf{X}}{\partial t}(\mathbf{s},0)=\mathbf{X}^{'}_0(\mathbf{s})&\forall\mathbf{s}\in\mathcal{B}
\end{align*}
La prima equazione del problema può essere separata in due equazioni del primo ordine
\begin{align*}
&M\frac{\partial \mathbf{U}}{\partial t}(t)+D^T\mathbb{P}(\mathbb{F}(t))=-\gamma_{\mathcal{B}}\mathbf{T}
\qquad&H^1(\mathcal{B}^d) \\
&M\frac{\partial \mathbf{X}}{\partial t}(t)-M\mathbf{U}(t)=0
\qquad&H^1(\mathcal{B}^d) \\
&\mathbf{X}(\mathbf{s},0)=\mathbf{X}_0(\mathbf{s})\qquad&\forall\mathbf{s}\in\mathcal{B}\\
&\mathbf{U}(\mathbf{s},0)=\mathbf{X}^{'}_0(\mathbf{s})&\forall\mathbf{s}\in\mathcal{B}
\end{align*}
dove
\begin{equation*}
\mathbf{U}(t) = \frac{\partial \mathbf{X}}{\partial t}(t)
\end{equation*}
è la velocità materiale.

\paragraph{Stima dell'energia}

Per $t\in (0,T)$ sia $\mathbf{X}(t)$ la soluzione del problema con $\mathbf{B}=0$ e $\mathbf{T}=0$ allora
\begin{equation*}
\frac{1}{2}\frac{d}{dt}\Vert\mathbf{U}(t)\Vert^2_{0,\mathcal{B}}+\frac{d}{dt}E[\mathbf{X}(t)]=0
\end{equation*}
\begin{proof}
Scegliendo
\begin{equation*}
\mathbf{Y}=\frac{\partial \mathbf{X}}{\partial t}=\mathbf{U}
\end{equation*}
allora l'equazione diventa
\begin{equation*}
\frac{1}{2}\frac{d}{dt}\ps{\frac{\partial\mathbf{X}}{\partial t}}{\frac{\partial \mathbf{X}}{\partial t}} + \ps{\mathbb{P}(\mathbb{F}(t))}{\frac{\partial}{\partial t} \Nabla_\mathbf{s} \mathbf{X}} = 0
\end{equation*}
\begin{equation*}
\frac{1}{2}\frac{d}{dt}\Vert\mathbf{U}\Vert^2 +\int\frac{\partial}{\partial t}\mathbb{W}\end{equation*}
\end{proof}
